% The Slide Definitions
%document
\documentclass[10pt]{beamer}
%theme
\usetheme{metropolis}
% packages
\usepackage{color}
\usepackage{listings}
\usepackage[ngerman]{babel}
\usepackage[utf8]{inputenc}
\usepackage{multicol}


% color definitions
\definecolor{mygreen}{rgb}{0,0.6,0}
\definecolor{mygray}{rgb}{0.5,0.5,0.5}
\definecolor{mymauve}{rgb}{0.58,0,0.82}

\lstset{
    backgroundcolor=\color{white},
    % choose the background color;
    % you must add \usepackage{color} or \usepackage{xcolor}
    basicstyle=\footnotesize\ttfamily,
    % the size of the fonts that are used for the code
    breakatwhitespace=false,
    % sets if automatic breaks should only happen at whitespace
    breaklines=true,                 % sets automatic line breaking
    captionpos=b,                    % sets the caption-position to bottom
    commentstyle=\color{mygreen},    % comment style
    % deletekeywords={...},
    % if you want to delete keywords from the given language
    extendedchars=true,
    % lets you use non-ASCII characters;
    % for 8-bits encodings only, does not work with UTF-8
    literate={ä}{{\"a}}1 {ü}{{\"u}}1 {ö}{{\"o}}1 {Ä}{{\"A}}1 {Ü}{{\"U}}1 {Ö}{{\"O}}1 {ß}{{\ss{}}}1,
    % escapes umlauts
    frame=single,                    % adds a frame around the code
    keepspaces=true,
    % keeps spaces in text,
    % useful for keeping indentation of code
    % (possibly needs columns=flexible)
    keywordstyle=\color{blue},       % keyword style
    % morekeywords={*,...},
    % if you want to add more keywords to the set
    numbers=left,
    % where to put the line-numbers; possible values are (none, left, right)
    numbersep=5pt,
    % how far the line-numbers are from the code
    numberstyle=\tiny\color{mygray},
    % the style that is used for the line-numbers
    rulecolor=\color{black},
    % if not set, the frame-color may be changed on line-breaks
    % within not-black text (e.g. comments (green here))
    stepnumber=1,
    % the step between two line-numbers.
    % If it's 1, each line will be numbered
    stringstyle=\color{mymauve},     % string literal style
    tabsize=4,                       % sets default tabsize to 4 spaces
    % show the filename of files included with \lstinputlisting;
    % also try caption instead of title
    language = Python,
	showspaces = false,
	showtabs = false,
	showstringspaces = false,
	escapechar = ,
}

\def\ContinueLineNumber{\lstset{firstnumber=last}}
\def\StartLineAt#1{\lstset{firstnumber=#1}}
\let\numberLineAt\StartLineAt



\newcommand{\codeline}[1]{
	\alert{\texttt{#1}}
}


% Author and Course information
% This Document contains the information about this course.

% Authors of the slides
\author{Richard Müller, Tom Felber}

% Name of the Course
\institute{Python-Kurs}

% Fancy Logo 
\titlegraphic{\hfill\includegraphics[height=1.25cm]{../templates/fsr_logo_cropped}}



% Custom Bindings
% \newcommand{\codeline}[1]{
%	\alert{\texttt{#1}}
%}


% Presentation title
\title{Module und Erweiterte Iteration}
\date{2. Dezember 2021}

\begin{document}
	
\maketitle

\begin{frame}{Gliederung}
	\setbeamertemplate{section in toc}[sections numbered]
	\tableofcontents
\end{frame}

\section*{Gesamtübersicht}
\begin{frame}{Gesamtübersicht}
	\textbf{Themen der nächsten Stunden}
	\begin{itemize}
		\item Referenzen Erklärung
		\item  Klassen
		\item \alert{Imports}
		\item \alert{Nützliche Funktionen zur Iteration}
		\item Lambda
		\item File handeling
		\item Listcomprehension
		\item Unpacking
		\item Dekoratoren
	\end{itemize}
\end{frame}

\section{Wiederholung}
\begin{frame}{Wiederholung}
	\textbf{Beim letzten Mal:}
	\centering\begin{itemize}
		\item Referenzen
		\item Klassen 
		\lstinputlisting[firstline=0,lastline=3]{resources/05mutation_klassen/einleitung_klassen.py}
	\end{itemize}

		
\end{frame}

\section{Module}
\begin{frame}{Grundlegendes}
	Erinnerung an Importe:
	\lstinputlisting[firstline=1,lastline=2]{resources//07imports_iteration/imp.py}
	oder:
	\lstinputlisting[firstline=4,lastline=5]{resources//07imports_iteration/imp.py}
	Importe funktionieren aus dem selben und aus niedrigeren Verzeichnissen ohne Probleme. Will man etwas aus einem höheren Verzeichniss importieren, so wird deutlich mehr Konfigurationsaufwand benötigt. Moderne IDEs nehmen das allerdings ab.	
\end{frame}

\begin{frame}{\_\_name\_\_ == '\_\_main\_\_'}
	Importiert man ein Modul, so wird jeder Code sofort ausgeführt, der nicht in Funktionen oder Klassen verpackt ist.
	
	Um das zu verhindern, gibt es ein 'Protection Statement':
	\lstinputlisting[firstline=7,lastline=8]{resources//07imports_iteration/imp.py}
	Die \codeline{\_\_name\_\_}-Variable nimmt nur den Wert \codeline{'\_\_main\_\_'} an, wenn das Modul direkt ausgeführt wird. 
	
	Wird es nur importiert, so hat die Variable einen anderen Wert. Dadurch wird das, was durch die \codeline{if}-Abfrage geschützt wird, nicht ausgeführt.
\end{frame}

\begin{frame}{Standardbibliothek}
	Die Standardbibliothek ist eine Sammlung von Modulen, die bereits nach der Installation von Python vorhanden ist.
	
	\textbf{Zwei Beispiele}
	
	\codeline{time:}
	\lstinputlisting[firstline=10,lastline=12]{resources//07imports_iteration/imp.py}
	\codeline{random:}
	\lstinputlisting[firstline=14,lastline=16]{resources//07imports_iteration/imp.py}
	In der \alert{\href{https://docs.python.org/3/library/}{Doku}} findet man alle verfügbaren Module.
\end{frame}

%\section{Nützliche Funktionen zur Iteration}
%\begin{frame}{enumerate}
%	Nimmt eine Liste entgegen und gibt sie als Tupelliste mit den jeweiligen Positionen in der Liste aus.
%	\begin{itemize}
%		\centering\item[Eingabe:] \codeline{['A', 'B', 'C', 'D']}
%		\centering\item[Ausgabe:] \codeline{[(0, 'A'), (1, 'B'), (2, 'C'), (3, 'D')]}
%	\end{itemize}
%\end{frame}
%\begin{frame}{enumerate}
%	Durch die Liste von Tupeln kann normal (tupelweise) iteriert werden.
%	\lstinputlisting[firstline=0,lastline=1]{resources//07imports_iteration/iteration.py}
%	\lstinputlisting[firstline=9,lastline=12]{resources//07imports_iteration/iteration.py}
%	Aber auch indem man den Elementen der jeweiligen Tupel direkt namen zuweist.
%	\lstinputlisting[firstline=3,lastline=6]{resources//07imports_iteration/iteration.py}
%\end{frame}
%
%\begin{frame}{zip}
%	Nimmt zwei (oder mehr) Listen entgegen und gibt sie als Tupelliste zurück, wobei Elemente an gleicher Position in der Liste in einem Tupel zusammengefasst werden.
%	\begin{itemize}
%		\centering\item[Eingabe:] \codeline{['A', 'B', 'C', 'D'], ['a', 'b', 'c', 'd']}
%		\centering\item[Ausgabe:] \codeline{[('A', 'a'), ('B', 'b'), ('C', 'c'), ('D', 'd')]}
%	\end{itemize}
%	\textbf{Hinweis:} Die Ausgabeliste ist immer so lang wie die kürzeste eingegebene Liste. Elemente von längeren Listen werden ignoriert.
%\end{frame}
%\begin{frame}{zip}
%	Wie bei enumerate kann durch die tuple,
%	\lstinputlisting[firstline=15,lastline=16]{resources//07imports_iteration/iteration.py}
%	\lstinputlisting[firstline=23,lastline=26]{resources//07imports_iteration/iteration.py}
%	oder direkt durch die Elemente iteriert werden.
%	\lstinputlisting[firstline=17,lastline=20]{resources//07imports_iteration/iteration.py}
%\end{frame}
%\begin{frame}{zip}
%	\textbf{Beispiel mit mehr als zwei Listen:}
%	\lstinputlisting[firstline=30,lastline=36]{resources//07imports_iteration/iteration.py}
%\end{frame}

\section{Keyword-Arguments}
\begin{frame}{Keyword-Arguments}
	Wenn man die Reihenfolge von Argumenten in einer Funktion übergehen will, kann man den Namen des Parameters direkt benutzen. Diese Argemente werden dann als \textbf{Keyword-Arguments} bezeichnet.
	\lstinputlisting[firstline=0,lastline=2]{resources//07imports_iteration/keyword.py}
	
	\begin{itemize}
		\item[\textbf{Ziel:}] \codeline{X Y Z}
		\lstinputlisting[firstline=4,lastline=4]{resources//07imports_iteration/keyword.py}
		\item[\textbf{Ziel:}] \codeline{X Y Hallo}
		\lstinputlisting[firstline=5,lastline=5]{resources//07imports_iteration/keyword.py}
		besser mit Keyword-Arguments
		\lstinputlisting[firstline=6,lastline=6]{resources//07imports_iteration/keyword.py}
	\end{itemize}
\end{frame}

\begin{frame}{Keyword-Arguments}
	Es kann auch erzwungen werden, dass Argumente Keyword-Arguments sind. Alle Argumente nach \codeline{, *, } sind zwangsweise Keyword-Arguments.
	\linebreak
	\lstinputlisting[firstline=8,lastline=9]{resources//07imports_iteration/keyword.py}
	\begin{itemize}
		\item[\textbf{Ziel:}] \codeline{X Y Z}
		\lstinputlisting[firstline=11,lastline=11]{resources//07imports_iteration/keyword.py}
	\end{itemize}
\end{frame}

\section{Itertools - Hilfe bei der Iteration}
\begin{frame}{Itertools}
	Itertools ist eine Bibliothek um die Iteration zu erleichtern. Unter anderem kann sie effizient große Iterationen durchführen.
	
	\lstinputlisting[firstline=1,lastline=1]{resources//07imports_iteration/itertools.py}
	
	\textbf{Beispiel:}
	\lstinputlisting[firstline=4,lastline=4]{resources//07imports_iteration/itertools.py}
	\lstinputlisting[firstline=8,lastline=12]{resources//07imports_iteration/itertools.py}
	Mit Itertools:
	\lstinputlisting[firstline=5,lastline=6]{resources//07imports_iteration/itertools.py}
\end{frame}

\begin{frame}{Itertools}
	\lstinputlisting[firstline=4,lastline=4]{resources//07imports_iteration/itertools.py}
	\lstinputlisting[firstline=18,lastline=22]{resources//07imports_iteration/itertools.py}
	Mit Itertools:
	\lstinputlisting[firstline=15,lastline=16]{resources//07imports_iteration/itertools.py}
\end{frame}

\end{document}
