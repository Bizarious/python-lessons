% The Slide Definitions
\input{../templates/course_definitions}

% Author and Course information
% This Document contains the information about this course.

% Authors of the slides
\author{Richard Müller, Tom Felber}

% Name of the Course
\institute{Python-Kurs}

% Fancy Logo 
\titlegraphic{\hfill\includegraphics[height=1.25cm]{../templates/fsr_logo_cropped}}



% Custom Bindings
% \newcommand{\codeline}[1]{
%	\alert{\texttt{#1}}
%}


% Presentation title
\title{Module und erweiterte Iteration}
\date{2. Dezember 2021}

\begin{document}
	
\maketitle

\begin{frame}{Gliederung}
	\setbeamertemplate{section in toc}[sections numbered]
	\tableofcontents
\end{frame}

\section*{Gesamtübersicht}
\begin{frame}{Gesamtübersicht}
	\textbf{Themen der nächsten Stunden}
	\begin{itemize}
		\item Referenzen Erklärung
		\item  Klassen
		\item \alert{Imports}
		\item \alert{Nützliche Funktionen zur Iteration}
		\item Lambda
		\item File handeling
		\item Listcomprehension
		\item Unpacking
		\item Dekoratoren
	\end{itemize}
\end{frame}

\section{Wiederholung}
\begin{frame}{Wiederholung}
	\textbf{Beim letzten Mal:}
	\centering\begin{itemize}
		\item Referenzen
		\item Klassen 
		\lstinputlisting[firstline=0,lastline=3]{resources/05mutation_klassen/einleitung_klassen.py}
	\end{itemize}

		
\end{frame}

\section{Module}
\begin{frame}{Module - Erinnerung}
\end{frame}

\section{Nützliche Funktionen zur Iteration}
\begin{frame}{enumerate}
	Nimmt eine Liste entgegen und gibt sie als Tupelliste mit den jeweiligen Positionen in der Liste aus.
	\begin{itemize}
		\centering\item[Eingabe:] \codeline{['A', 'B', 'C', 'D']}
		\centering\item[Ausgabe:] \codeline{[(0, 'A'), (1, 'B'), (2, 'C'), (3, 'D')]}
	\end{itemize}
\end{frame}
\begin{frame}{enumerate}
	Durch die Liste von Tupeln kann normal (tupelweise) iteriert werden.
	\lstinputlisting[firstline=0,lastline=1]{resources//07imports_iteration/iteration.py}
	\lstinputlisting[firstline=9,lastline=12]{resources//07imports_iteration/iteration.py}
	Aber auch indem man den Elementen der jeweiligen Tupel direkt namen zuweist.
	\lstinputlisting[firstline=3,lastline=6]{resources//07imports_iteration/iteration.py}
\end{frame}

\begin{frame}{zip}
	Nimmt zwei (oder mehr) Listen entgegen und gibt sie als Tupelliste zurück, wobei Elemente an gleicher Position in der Liste in einem Tupel zusammengefasst werden.
	\begin{itemize}
		\centering\item[Eingabe:] \codeline{['A', 'B', 'C', 'D'], ['a', 'b', 'c', 'd']}
		\centering\item[Ausgabe:] \codeline{[('A', 'a'), ('B', 'b'), ('C', 'c'), ('D', 'd')]}
	\end{itemize}
	\textbf{Hinweis:} Die Ausgabeliste ist immer so lang wie die kürzeste eingegebene Liste. Elemente von längeren Listen werden ignoriert.
\end{frame}
\begin{frame}{zip}
	Wie bei enumerate kann durch die tuple,
	\lstinputlisting[firstline=15,lastline=16]{resources//07imports_iteration/iteration.py}
	\lstinputlisting[firstline=23,lastline=26]{resources//07imports_iteration/iteration.py}
	oder direkt durch die Elemente iteriert werden.
	\lstinputlisting[firstline=17,lastline=20]{resources//07imports_iteration/iteration.py}
\end{frame}
\begin{frame}{zip}
	\textbf{Beispiel mit mehr als zwei Listen:}
	\lstinputlisting[firstline=30,lastline=36]{resources//07imports_iteration/iteration.py}
\end{frame}

\end{document}
