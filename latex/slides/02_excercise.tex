% The Slide Definitions
\input{../templates/course_definitions}

% Author and Course information
% This Document contains the information about this course.

% Authors of the slides
\author{Richard Müller, Tom Felber}

% Name of the Course
\institute{Python-Kurs}

% Fancy Logo 
\titlegraphic{\hfill\includegraphics[height=1.25cm]{../templates/fsr_logo_cropped}}



% Custom Bindings
% \newcommand{\codeline}[1]{
%	\alert{\texttt{#1}}
%}


% Presentation title
% TODO Change the topic of the lesson
\title{Zahlen, Listen und Schleifen - Übung}
\date{28. Oktober 2021}
\begin{document}

\maketitle

\begin{frame}{Gliederung}
	\setbeamertemplate{section in toc}[sections numbered]
	\tableofcontents
\end{frame}

% TODO: Add your content right below here.

\section{Übung 1 - Zahlen und Listen}
\begin{frame}{Übung 1}
	Schreibe ein Programm, welches folgendes tut:
	\linebreak
	
	\begin{itemize}
		\item[\textbf{1.}] Erstelle eine leere Liste und speicher diese in einer Variable.
		\item[\textbf{2.}] Fülle die Liste nacheinander mit den Zahlen 13, 7 und 0.
		\item[\textbf{3.}] Addiere das erste und zweite Listenelement und überschreibe  mit dem Ergebnis das dritte.
		\item[\textbf{4.}] Entferne das zweite Element und hänge es hinten an die Liste wieder an.
		\item[\textbf{Zusatz}] Prüfe, ob das erste Element kleiner ist als das zweite. Wenn das so ist, füge hinten an die Liste \alert{\texttt{True}} an, ansonsten \alert{\texttt{False}}.
	\end{itemize}
	
\end{frame}

\end{document}
