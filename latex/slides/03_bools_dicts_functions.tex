% The Slide Definitions
\input{../templates/course_definitions}

% Author and Course information
% This Document contains the information about this course.

% Authors of the slides
\author{Richard Müller, Tom Felber}

% Name of the Course
\institute{Python-Kurs}

% Fancy Logo 
\titlegraphic{\hfill\includegraphics[height=1.25cm]{../templates/fsr_logo_cropped}}



% Custom Bindings
% \newcommand{\codeline}[1]{
%	\alert{\texttt{#1}}
%}


% Presentation title
\title{Booleans und Funktionen}
\date{4. November 2021}

\usepackage{graphicx}
\usepackage{array}

\begin{document}
	
\maketitle

\begin{frame}{Gliederung}
	\setbeamertemplate{section in toc}[sections numbered]
	\tableofcontents
\end{frame}


\section{Wiederholung}
\begin{frame}{Wiederholung}
	\textbf{Beim letzten Mal}
	\begin{itemize}
		\item Zahlen
		\item Listen
		\lstinputlisting[firstline=1,lastline=1]{resources/03bool_fun_dict/rep.py}
		\item \alert{\texttt{for}}-Schleife
		\lstinputlisting[firstline=3,lastline=4]{resources/03bool_fun_dict/rep.py}
		\item \alert{\texttt{while}}-Schleife
		\lstinputlisting[firstline=6,lastline=9]{resources/03bool_fun_dict/rep.py}
	\end{itemize}
\end{frame}

\section{Booleans}
\begin{frame}{Booleans}
	Booleans können die Werte \alert{\texttt{True}} oder \alert{\texttt{False}} annehmen.
	\linebreak
	Die Keywords \alert{\texttt{and}}, \alert{\texttt{or}} können benutzt werden um Booleans logisch zu verbinden.
	\linebreak
	Mit \alert{\texttt{not}} kann ein Boolean invertiert werden:

	\begin{center}
	\begin{tabular}{c | c}
		\hline\hline
		mit Operator & Wert \\
		\hline\hline
		\texttt{True and False} & False \\
		\texttt{True and True} & True \\
		\texttt{True or False} & True \\
		\texttt{False or False} & False \\
		\hline\hline
		\texttt{not True} & False \\
		\texttt{not False} & True \\
		\texttt{not not True} & True
	\end{tabular}
	\end{center}
\end{frame}

\begin{frame}{Booleans}
	Kombinationen der Operatoren sind auch möglich. Dabei können Klammern gesetzt werden um die Reihenfolge der Auswertung zu bestimmen. Wenn keine gesetzt werden, wird \alert{\texttt{and}} vor \alert{\texttt{or}} ausgewertet. \alert{\texttt{not}} wird zuallererst ausgewertet.
	\begin{center}
		\begin{tabular}{m{8cm} | m{1cm}}
			\hline\hline
			mit Operator & Wert \\
			\hline\hline
			\texttt{False and True or True} & True \\
			\texttt{False and (True or True)} & False \\
			\texttt{(not not not False and True and not True) or (True and True)} & ?
		\end{tabular}
	\end{center}
\end{frame}


\section{Funktionen}
\begin{frame}{Funktionen - Einleitung}
		\Large $$ f(x) = 3x^5 + 8x^4 + 42x^3 + x $$
\end{frame}

\begin{frame}{Funktionen - Einleitung}
	Funktionen sind immer dann gut geeignet, wenn man den selben Code an mehreren Stellen ausführen muss.
		
	Zum Beispiel ist es einfacher, \alert{\texttt{print()}} zu benutzen, als jedes Mal den kompletten Code dieser Funktion in sein eigenes Programm zu kopieren.
\end{frame}

\begin{frame}{Funktionen - Aufbau}
	Die einfachst mögliche Funktion könnte so aussehen:
	\lstinputlisting[lastline=2]{resources/03bool_fun_dict/func.py}
	\pause
	Jede Funktion besteht aus folgenden Teilen:
	\pause
	\begin{itemize}
		\item dem Schlüsselwort \alert{\texttt{def}}
		\pause
		\item dem Name der Funktion - grundsätzlich beliebig, aber man sollte  die Konvention beachten (alles klein, mehrere Worte mit Unterstrich trennen)
		\pause
		\item zwei Runde Klammern, in denen Argumente definiert werden können
		\pause
		\item dem Code, der ausgeführt werden soll, eingerückt unter dem Funktionskopf. \textbf{Man kann in einer Funktion alles machen, was man auch außerhalb kann.}
	\end{itemize}
\end{frame}

\begin{frame}{Funktionen - Aufruf}
	Der Code innerhalb einer Funktion wird nicht sofort ausgeführt, wenn die Funktion definiert wird, stattdessen muss man sie explizit aufrufen:
	\lstinputlisting[firstline=4, lastline=5]{resources/03bool_fun_dict/func.py}
\end{frame}

\begin{frame}{Funktionen - Argumente}
	In den Klammern können Argumente festgelegt werden, die dann beim Aufruf mit Werten belegt werden müssen:
	\lstinputlisting[firstline=7, lastline=12]{resources/03bool_fun_dict/func.py}
\end{frame}

\begin{frame}{Funktionen - Return}
	Möchte man einen Wert zurückgeben, ihn also auch außerhalb der Funktion noch verwenden, kann man das mit dem \alert{\texttt{return}}-Statement tun:
	\lstinputlisting[firstline=14, lastline=18]{resources/03bool_fun_dict/func.py}
	Das \alert{\texttt{return}}-Statement bricht die Funktion ab. Alles was danach noch käme, wird nicht mehr ausgeführt.
\end{frame}

\begin{frame}{Funktionen - Default Argumente}
	Man kann Argumenten Default-Werte mitgeben. Somit wird dieses Argument optional und muss nicht zwingend während des Aufrufs belegt werden. Gibt man ihm trotzdem einen Wert, so wird der Default-Wert überschrieben:
	\lstinputlisting[firstline=20, lastline=24]{resources/03bool_fun_dict/func.py}
	Default Argumente müssen in den Klammern der Funktionsdefinition immer als letztes angegeben werden.
\end{frame}

\end{document}
