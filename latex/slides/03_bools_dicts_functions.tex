% The Slide Definitions
\input{../templates/course_definitions}

% Author and Course information
% This Document contains the information about this course.

% Authors of the slides
\author{Richard Müller, Tom Felber}

% Name of the Course
\institute{Python-Kurs}

% Fancy Logo 
\titlegraphic{\hfill\includegraphics[height=1.25cm]{../templates/fsr_logo_cropped}}



% Custom Bindings
% \newcommand{\codeline}[1]{
%	\alert{\texttt{#1}}
%}


% Presentation title
\title{Booleans, Funktionen und Dicionaries}
\date{28. Oktober 2021}

\usepackage{graphicx}

\begin{document}
	
\maketitle

\begin{frame}{Gliederung}
	\setbeamertemplate{section in toc}[sections numbered]
	\tableofcontents
\end{frame}


\section{Wiederholung}
\begin{frame}{Wiederholung}
	test
\end{frame}

\section{Booleans}
\begin{frame}{Booleans}
	test	
\end{frame}

\section{Funktionen}
\begin{frame}{Funktionen}
	test
\end{frame}

\section{Dictionaries}
\begin{frame}{Dictionaries - Einleitung}
	\includegraphics[width=\linewidth]{resources/03bool_fun_dict/lexikon.jpeg}
\end{frame}
\begin{frame}{Dictionaries - Erstellen / Lesen}
	In Dictionaries können \alert{\texttt{Key - Value}} Paare abgelegt werden.
	\linebreak
	Die Keys müssen dabei einzigartig sein. Die Values nicht.
	\linebreak
	Keys werden rechts, Values links angegeben.
	\lstinputlisting[lastline=4]{resources/03bool_fun_dict/dicts.py}
	Mit dem Key kann der zugehörige Wert gelesen werden.
	\lstinputlisting[firstline=6,lastline=7]{resources/03bool_fun_dict/dicts.py}
\end{frame}
\begin{frame}{Dictionaries - Paare hinzufügen / überschreiben}
	Die Syntax zum hinzufügen kann genauso benutzt werden um einen vorhandenen Key zu überschreiben
	\lstinputlisting[firstline=4,lastline=4]{resources/03bool_fun_dict/dicts.py}
	\lstinputlisting[firstline=8,lastline=11]{resources/03bool_fun_dict/dicts.py}
	Die meisten Typen können als Key benutzt werden. Listen und Dictionaries können nicht als Key benutzt werden. Warum das so ist dazu später mehr.
	\lstinputlisting[firstline=23,lastline=30]{resources/03bool_fun_dict/dicts.py}
\end{frame}

\end{document}
