% The Slide Definitions
\input{../templates/course_definitions}

% Author and Course information
% This Document contains the information about this course.

% Authors of the slides
\author{Richard Müller, Tom Felber}

% Name of the Course
\institute{Python-Kurs}

% Fancy Logo 
\titlegraphic{\hfill\includegraphics[height=1.25cm]{../templates/fsr_logo_cropped}}



% Custom Bindings
% \newcommand{\codeline}[1]{
%	\alert{\texttt{#1}}
%}


% Presentation title
\title{Booleans, Funktionen und Dicionaries}
\date{28. Oktober 2021}

\usepackage{graphicx}

\begin{document}
	
\maketitle

\begin{frame}{Gliederung}
	\setbeamertemplate{section in toc}[sections numbered]
	\tableofcontents
\end{frame}


\section{Wiederholung}
\begin{frame}{Wiederholung}
	test
\end{frame}

\section{Booleans}
\begin{frame}{Booleans}
	test	
\end{frame}

\section{Funktionen}
\begin{frame}{Funktionen - Einleitung}
		\Large $$ f(x) = 3x^5 + 8x^4 + 42x^3 + x $$
\end{frame}

\begin{frame}{Funktionen - Einleitung}
	Funktionen sind immer dann gut geeignet, wenn man den selben Code an mehreren Stellen ausführen muss.
		
	Zum Beispiel ist es einfacher, \alert{\texttt{print()}} zu benutzen, als jedes Mal den kompletten Code dieser Funktion in sein eigenes Programm zu kopieren.
\end{frame}

\begin{frame}{Funktionen - Aufbau}
	Die einfachst mögliche Funktion könnte so aussehen:
	\lstinputlisting[lastline=2]{resources/03bool_fun_dict/func.py}
	\pause
	Jede Funktion besteht aus folgenden Teilen:
	\pause
	\begin{itemize}
		\item dem Schlüsselwort \alert{\texttt{def}}
		\pause
		\item dem Name der Funktion - grundsätzlich beliebig, aber man sollte  die Konvention beachten (alles klein, mehrere Worte mit Unterstrich trennen)
		\pause
		\item zwei Runde Klammern, in denen Argumente definiert werden können
		\pause
		\item dem Code, der ausgeführt werden soll, eingerückt unter dem Funktionskopf. \textbf{Man kann in einer Funktion alles machen, was man auch außerhalb kann.}
	\end{itemize}
\end{frame}

\begin{frame}{Funktionen - Aufruf}
	Der Code innerhalb einer Funktion wird nicht sofort ausgeführt, wenn die Funktion definiert wird, stattdessen muss man sie explizit aufrufen:
	\lstinputlisting[firstline=4, lastline=5]{resources/03bool_fun_dict/func.py}
\end{frame}

\begin{frame}{Funktionen - Argumente}
	In den Klammern können Argumente festgelegt werden, die dann beim Aufruf mit Werten belegt werden müssen:
	\lstinputlisting[firstline=7, lastline=12]{resources/03bool_fun_dict/func.py}
\end{frame}

\begin{frame}{Funktionen - Return}
	Möchte man einen Wert zurückgeben, ihn also auch außerhalb der Funktion noch verwenden, kann man das mit dem \alert{\texttt{return}}-Statement tun:
	\lstinputlisting[firstline=14, lastline=18]{resources/03bool_fun_dict/func.py}
\end{frame}

\section{Dictionaries}
\begin{frame}{Dictionaries - Einleitung}
	\includegraphics[width=\linewidth]{resources/03bool_fun_dict/lexikon.jpeg}
\end{frame}
\begin{frame}{Dictionaries - Erstellen / Lesen}
	In Dictionaries können \alert{\texttt{Key - Value}} Paare abgelegt werden.
	\linebreak
	Die Keys müssen dabei einzigartig sein. Die Values nicht.
	\linebreak
	Keys werden links, Values rechts angegeben.
	\lstinputlisting[lastline=4]{resources/03bool_fun_dict/dicts.py}
	Mit dem Key kann der zugehörige Wert gelesen werden.
	\lstinputlisting[firstline=6,lastline=7]{resources/03bool_fun_dict/dicts.py}
\end{frame}
\begin{frame}{Dictionaries - Paare hinzufügen / überschreiben}
	Die Syntax zum hinzufügen kann genauso benutzt werden um einen vorhandenen Key zu überschreiben
	\lstinputlisting[firstline=4,lastline=4]{resources/03bool_fun_dict/dicts.py}
	\lstinputlisting[firstline=8,lastline=11]{resources/03bool_fun_dict/dicts.py}
	Die meisten Typen können als Key benutzt werden. Listen und Dictionaries können nicht als Key benutzt werden. Warum das so ist dazu später mehr.
	\lstinputlisting[firstline=23,lastline=30]{resources/03bool_fun_dict/dicts.py}
\end{frame}

\end{document}
