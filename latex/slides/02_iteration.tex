% The Slide Definitions
\input{../templates/course_definitions}

% Author and Course information
% This Document contains the information about this course.

% Authors of the slides
\author{Richard Müller, Tom Felber}

% Name of the Course
\institute{Python-Kurs}

% Fancy Logo 
\titlegraphic{\hfill\includegraphics[height=1.25cm]{../templates/fsr_logo_cropped}}



% Custom Bindings
% \newcommand{\codeline}[1]{
%	\alert{\texttt{#1}}
%}


% Presentation title
\title{Iteration Übung}
\date{28. Oktober 2021}


\begin{document}
	
\maketitle

\begin{frame}{Gliederung}
	\setbeamertemplate{section in toc}[sections numbered]
	\tableofcontents
\end{frame}


\section{Listen}
\begin{frame}{Listen}
	test
\end{frame}

\section{for in Schleife}
\begin{frame}{for in Schleife}
	die Zahlen von 0 bis 999 durchgehen:
	\lstinputlisting[firstline=0,lastline=3]{resources/02iteration/for_loop.py}
	Objekte, die 'iterable' sind, elementweise durchgehen:
	\lstinputlisting[firstline=5,lastline=8]{resources/02iteration/for_loop.py}
	\lstinputlisting[firstline=10,lastline=13]{resources/02iteration/for_loop.py}
\end{frame}

\section{while Schleife}
	\begin{frame}{while Schleife}
		test
	\end{frame}
\end{document}
