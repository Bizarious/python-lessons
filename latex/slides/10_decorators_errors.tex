% The Slide Definitions
%document
\documentclass[10pt]{beamer}
%theme
\usetheme{metropolis}
% packages
\usepackage{color}
\usepackage{listings}
\usepackage[ngerman]{babel}
\usepackage[utf8]{inputenc}
\usepackage{multicol}


% color definitions
\definecolor{mygreen}{rgb}{0,0.6,0}
\definecolor{mygray}{rgb}{0.5,0.5,0.5}
\definecolor{mymauve}{rgb}{0.58,0,0.82}

\lstset{
    backgroundcolor=\color{white},
    % choose the background color;
    % you must add \usepackage{color} or \usepackage{xcolor}
    basicstyle=\footnotesize\ttfamily,
    % the size of the fonts that are used for the code
    breakatwhitespace=false,
    % sets if automatic breaks should only happen at whitespace
    breaklines=true,                 % sets automatic line breaking
    captionpos=b,                    % sets the caption-position to bottom
    commentstyle=\color{mygreen},    % comment style
    % deletekeywords={...},
    % if you want to delete keywords from the given language
    extendedchars=true,
    % lets you use non-ASCII characters;
    % for 8-bits encodings only, does not work with UTF-8
    literate={ä}{{\"a}}1 {ü}{{\"u}}1 {ö}{{\"o}}1 {Ä}{{\"A}}1 {Ü}{{\"U}}1 {Ö}{{\"O}}1 {ß}{{\ss{}}}1,
    % escapes umlauts
    frame=single,                    % adds a frame around the code
    keepspaces=true,
    % keeps spaces in text,
    % useful for keeping indentation of code
    % (possibly needs columns=flexible)
    keywordstyle=\color{blue},       % keyword style
    % morekeywords={*,...},
    % if you want to add more keywords to the set
    numbers=left,
    % where to put the line-numbers; possible values are (none, left, right)
    numbersep=5pt,
    % how far the line-numbers are from the code
    numberstyle=\tiny\color{mygray},
    % the style that is used for the line-numbers
    rulecolor=\color{black},
    % if not set, the frame-color may be changed on line-breaks
    % within not-black text (e.g. comments (green here))
    stepnumber=1,
    % the step between two line-numbers.
    % If it's 1, each line will be numbered
    stringstyle=\color{mymauve},     % string literal style
    tabsize=4,                       % sets default tabsize to 4 spaces
    % show the filename of files included with \lstinputlisting;
    % also try caption instead of title
    language = Python,
	showspaces = false,
	showtabs = false,
	showstringspaces = false,
	escapechar = ,
}

\def\ContinueLineNumber{\lstset{firstnumber=last}}
\def\StartLineAt#1{\lstset{firstnumber=#1}}
\let\numberLineAt\StartLineAt



\newcommand{\codeline}[1]{
	\alert{\texttt{#1}}
}


% Author and Course information
% This Document contains the information about this course.

% Authors of the slides
\author{Richard Müller, Tom Felber}

% Name of the Course
\institute{Python-Kurs}

% Fancy Logo 
\titlegraphic{\hfill\includegraphics[height=1.25cm]{../templates/fsr_logo_cropped}}



% Custom Bindings
% \newcommand{\codeline}[1]{
%	\alert{\texttt{#1}}
%}


% Presentation title
\title{Dekoratoren und Exception-Behandlung}
\date{20. Januar 2022}

\begin{document}
	
\maketitle

\begin{frame}{Gliederung}
	\setbeamertemplate{section in toc}[sections numbered]
	\tableofcontents
\end{frame}

\section*{Gesamtübersicht}
\begin{frame}{Gesamtübersicht}
	\textbf{Themen der nächsten Stunden}
	\begin{itemize}
		\item Referenzen Erklärung
		\item  Klassen
		\item Imports
		\item Nützliche Funktionen zur Iteration
		\item Lambda
		\item Unpacking
		\item File handeling
		\item Listcomprehension
		\item \alert{Dekoratoren}
		\item \alert{Exception-Behandlung}
	\end{itemize}
\end{frame}

\section{Wiederholung}
\begin{frame}{Wiederholung}
	\textbf{Beim letzten Mal:}
	\begin{itemize}
		\item Filehandling
		\lstinputlisting[firstline=3,lastline=8]{resources/09_filehandeling_listcomprehension/open_basics.py}
		\item List Comprehensions
		\lstinputlisting[firstline=0,lastline=7]{resources/09_filehandeling_listcomprehension/comprehensions.py}
	\end{itemize}	
\end{frame}

\section{Dekoratoren}
\begin{frame}{Dekoratoren}
	Funktionen und Klassen Definitionen können "dekoriert" werden, indem über der jeweiligen definition \codeline{@<funktion>} eingefügt wird.
	\lstinputlisting[firstline=8,lastline=10]{resources/10_decorators_errors/dekoratoren.py}
	Hinter dem \codeline{@} muss dabei ein Funktionsname stehen.	
\end{frame}

\begin{frame}
	Funktionen die als Dekorator verwendet werden können, nehmen eine Funktion, Methode oder Klasse als Argument entgegen und geben üblicherweise das gleiche auch wieder aus. 
	
	Durch die \codeline{@} Notation wird die ursprüngliche Funktion mit dem Ausgabewert der Dekoratorfunktion überschrieben.
	\lstinputlisting[firstline=12,lastline=15]{resources/10_decorators_errors/dekoratoren.py}
	Dieser Code bewirkt das Gleiche, was auch mit der \codeline{@} Notation bewirkt würde. 
\end{frame}

\begin{frame}{Beispiel Funktionsdekorator}
	Ein Funktionsdekorator nimmt eine Funktion entgegen, und gibt meistens auch wieder eine Funktion aus.
	\lstinputlisting[firstline=2,lastline=6]{resources/10_decorators_errors/dekoratoren.py}
	\lstinputlisting[firstline=8,lastline=10]{resources/10_decorators_errors/dekoratoren.py}
	In diesem Fall verändert die Dekoratorfunktion die Signatur der Ausgangsfunktion. 
\end{frame}

\begin{frame}{Anwendungsbeispiel Nebeneffekte}
	Ziel eines Dekorators muss es aber nicht immer sein, das Verhalten der Funktion oder Klasse zu verändern. Er kann auch als Einstiegspunkt dienen.
	\lstinputlisting[firstline=18,lastline=31]{resources/10_decorators_errors/dekoratoren.py}
\end{frame}

\begin{frame}{mehrere Dekoratoren}
	Es können auch mehrere Dekoratoren angewendet werden:
	\lstinputlisting[firstline=54,lastline=59]{resources/10_decorators_errors/dekoratoren.py}
\end{frame}

\begin{frame}
	Definitionen für die benutzten Dekoratoren.
	\lstinputlisting[firstline=40,lastline=52]{resources/10_decorators_errors/dekoratoren.py}
\end{frame}

\begin{frame}{mehrere Dekoratoren}
	Es können auch mehrere Dekoratoren angewendet werden:
	\lstinputlisting[firstline=54,lastline=59]{resources/10_decorators_errors/dekoratoren.py}
	Ausgabe: \linebreak
	\codeline{deko eins start} \linebreak
	\codeline{deko zwei start} \linebreak
	\codeline{hi} \linebreak
	\codeline{deko zwei ende} \linebreak
	\codeline{deko eins ende}
\end{frame}

\section{Exception-Behandlung}
\begin{frame}{Exceptions}
	\alert{Exceptions} sind Fehler, die das Programm wirft, wenn es einen unerwünschten Zustand erreicht hat. Jedem Programmierer werden sie früher oder später begegnen.
	
	Beispiel Exceptions sind:
	\begin{itemize}
		\item \codeline{Exception}(Basisklasse, von der alle Exceptions erben)
		\item \codeline{RuntimeError}
		\item \codeline{ZeroDivisionError}
		\item \codeline{ValueError}
		\item \codeline{KeyboardInterrupt}
	\end{itemize}
\end{frame}

\begin{frame}{\codeline{try} und \codeline{except}}
	Mit den beiden Keywords \codeline{try} und \codeline{except} können Fehler abgefangen und behandelt werden:
	\lstinputlisting[firstline=0,lastline=5]{resources/10_decorators_errors/errors.py}
	Das Programm wird nicht abstürzen, es wird lediglich 'Da war ein Fehler' auf der Konsole ausgeben.
\end{frame}

\begin{frame}{\codeline{try} und \codeline{except}}
	Man kann, ähnlich wie bei \codeline{if} und \codeline{elif}, auch verschiedene Fehler verschieden behandeln:
	\lstinputlisting[firstline=7,lastline=14]{resources/10_decorators_errors/errors.py}
\end{frame}

\begin{frame}{\codeline{finally}}
	Möchte man \textit{auf jeden Fall} etwas ausführen, egal, ob gerade ein Fehler kam, oder nicht, kann man das Keyword \codeline{finally} benutzen:
	\lstinputlisting[firstline=28,lastline=34]{resources/10_decorators_errors/errors.py}
\end{frame}

\begin{frame}{Exceptions sind Objekte}
	Wird in Python eine Exception geworfen, so ist diese als Objekt verfügbar:
	\lstinputlisting[firstline=16,lastline=20]{resources/10_decorators_errors/errors.py}
	oder allgemeiner:
	\lstinputlisting[firstline=22,lastline=26]{resources/10_decorators_errors/errors.py}
\end{frame}

\begin{frame}{Selber Exceptions werfen}
	Man kann auch selber Exceptions werfen, z.B., wenn man gewisse Eingaben unterbinden will. Dies wird mit dem Keyword \codeline{raise} gemacht:
	\lstinputlisting[firstline=36,lastline=40]{resources/10_decorators_errors/errors.py}
\end{frame}

\begin{frame}{Ausblick auf nächste Stunde}
	\begin{itemize}
		\item Flask/Django
		\item Discord Bot
		\item Selenium
		\item Erweiterte Themen
		\begin{itemize}
			\item Multithreading
			\item Datetime
			\item Regex
			\item ...
		\end{itemize}
	\end{itemize}
\end{frame}

\end{document}
