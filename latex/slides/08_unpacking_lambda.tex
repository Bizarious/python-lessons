% The Slide Definitions
%document
\documentclass[10pt]{beamer}
%theme
\usetheme{metropolis}
% packages
\usepackage{color}
\usepackage{listings}
\usepackage[ngerman]{babel}
\usepackage[utf8]{inputenc}
\usepackage{multicol}


% color definitions
\definecolor{mygreen}{rgb}{0,0.6,0}
\definecolor{mygray}{rgb}{0.5,0.5,0.5}
\definecolor{mymauve}{rgb}{0.58,0,0.82}

\lstset{
    backgroundcolor=\color{white},
    % choose the background color;
    % you must add \usepackage{color} or \usepackage{xcolor}
    basicstyle=\footnotesize\ttfamily,
    % the size of the fonts that are used for the code
    breakatwhitespace=false,
    % sets if automatic breaks should only happen at whitespace
    breaklines=true,                 % sets automatic line breaking
    captionpos=b,                    % sets the caption-position to bottom
    commentstyle=\color{mygreen},    % comment style
    % deletekeywords={...},
    % if you want to delete keywords from the given language
    extendedchars=true,
    % lets you use non-ASCII characters;
    % for 8-bits encodings only, does not work with UTF-8
    literate={ä}{{\"a}}1 {ü}{{\"u}}1 {ö}{{\"o}}1 {Ä}{{\"A}}1 {Ü}{{\"U}}1 {Ö}{{\"O}}1 {ß}{{\ss{}}}1,
    % escapes umlauts
    frame=single,                    % adds a frame around the code
    keepspaces=true,
    % keeps spaces in text,
    % useful for keeping indentation of code
    % (possibly needs columns=flexible)
    keywordstyle=\color{blue},       % keyword style
    % morekeywords={*,...},
    % if you want to add more keywords to the set
    numbers=left,
    % where to put the line-numbers; possible values are (none, left, right)
    numbersep=5pt,
    % how far the line-numbers are from the code
    numberstyle=\tiny\color{mygray},
    % the style that is used for the line-numbers
    rulecolor=\color{black},
    % if not set, the frame-color may be changed on line-breaks
    % within not-black text (e.g. comments (green here))
    stepnumber=1,
    % the step between two line-numbers.
    % If it's 1, each line will be numbered
    stringstyle=\color{mymauve},     % string literal style
    tabsize=4,                       % sets default tabsize to 4 spaces
    % show the filename of files included with \lstinputlisting;
    % also try caption instead of title
    language = Python,
	showspaces = false,
	showtabs = false,
	showstringspaces = false,
	escapechar = ,
}

\def\ContinueLineNumber{\lstset{firstnumber=last}}
\def\StartLineAt#1{\lstset{firstnumber=#1}}
\let\numberLineAt\StartLineAt



\newcommand{\codeline}[1]{
	\alert{\texttt{#1}}
}


% Author and Course information
% This Document contains the information about this course.

% Authors of the slides
\author{Richard Müller, Tom Felber}

% Name of the Course
\institute{Python-Kurs}

% Fancy Logo 
\titlegraphic{\hfill\includegraphics[height=1.25cm]{../templates/fsr_logo_cropped}}



% Custom Bindings
% \newcommand{\codeline}[1]{
%	\alert{\texttt{#1}}
%}


% Presentation title
\title{Mehr zu Funktionen}
\date{9. Dezember 2021}

\begin{document}
	
\maketitle

\begin{frame}{Gliederung}
	\setbeamertemplate{section in toc}[sections numbered]
	\tableofcontents
\end{frame}

\section*{Gesamtübersicht}
\begin{frame}{Gesamtübersicht}
	\textbf{Themen der nächsten Stunden}
	\begin{itemize}
		\item Referenzen Erklärung
		\item  Klassen
		\item Imports
		\item Nützliche Funktionen zur Iteration
		\item \alert{Lambda}
		\item \alert{Unpacking}
		\item File handeling
		\item Listcomprehension
		\item Dekoratoren
	\end{itemize}
\end{frame}

\section{Wiederholung}
\begin{frame}{Wiederholung}
	\textbf{Beim letzten Mal:}
	\begin{itemize}
		\item Vererbung
		\lstinputlisting[firstline=9,lastline=10]{resources/06_inheritance_modules/inherit.py}
		\item Module Einführung
		\lstinputlisting[firstline=0,lastline=2]{resources/06_inheritance_modules/modules.py}
	\end{itemize}	
\end{frame}

\section{Unpacking}

\begin{frame}{Was ist Unpacking?}
	Unpacking, bzw. Packing, bezeichnet eine Operation, mit der man Elemente eines \codeline{iterables} direkt an Variablen binden kann. Wir kennen das bereits von Tupeln:
	\lstinputlisting[firstline=1,lastline=5]{resources//08_unpacking_lambda/unpacking.py}
	Das funktioniert mit jedem \codeline{iterable}, also auch mit Listen zum Beispiel.
\end{frame}

\begin{frame}{Der *-Operator}
	Der *-Operator gestattet das Zusammenfassen mehrerer Werte in einer einzigen Variable:
	\lstinputlisting[firstline=7,lastline=8]{resources//08_unpacking_lambda/unpacking.py}
	Auch wenn nach der Variable (hier \codeline{a}) nichts mehr folgt, muss trotzdem das Komma gesetzt werden.
\end{frame}

\begin{frame}{Der *-Operator}
	Man kann den *-Operator auch zusammen mit dem herkömmlichen Packing/Unpacking verbinden. Python ermittelt selbst, welche Variable welche Werte bekommen muss:
	\lstinputlisting[firstline=10,lastline=17]{resources//08_unpacking_lambda/unpacking.py}
	\begin{description}
		\item[Achtung] In so einem Ausdruck darf der *-Operator höchstens einmal vorkommen.
	\end{description}
\end{frame}

\begin{frame}{Der *-Operator}
	Man kann Unpacking auch in Verbindung mit Funktionen benutzen:
	\lstinputlisting[firstline=19,lastline=23]{resources//08_unpacking_lambda/unpacking.py}
	oder anders herum:
	\lstinputlisting[firstline=25,lastline=29]{resources//08_unpacking_lambda/unpacking.py}
\end{frame}

\begin{frame}{Der *-Operator}
	oder zusammen:
	\lstinputlisting[firstline=31,lastline=35]{resources//08_unpacking_lambda/unpacking.py}
	Hier wird das Tupel ausgepackt, sodass alle Elemente einzeln in die Funktion gegeben werden. Effektiv verhält es sich also wie in Beispiel 1.
\end{frame}

\begin{frame}{Der **-Operator}
	Der **-Operator ist das Unpacking-Equivalent für Dictionarys. Dieser ist vor allem für Funktionen interessant, denn durch ihn kann man das Unpacking mit Keyword-Argumenten verbinden:
	\lstinputlisting[firstline=37,lastline=41]{resources//08_unpacking_lambda/unpacking.py}
	oder zusammen mit dem *-Operator:
	\lstinputlisting[firstline=43,lastline=49]{resources//08_unpacking_lambda/unpacking.py}
\end{frame}



\section{Lamdba}

\begin{frame}{}

\end{frame}

\end{document}
