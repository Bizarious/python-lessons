% The Slide Definitions
\input{../templates/course_definitions}

% Author and Course information
% This Document contains the information about this course.

% Authors of the slides
\author{Richard Müller, Tom Felber}

% Name of the Course
\institute{Python-Kurs}

% Fancy Logo 
\titlegraphic{\hfill\includegraphics[height=1.25cm]{../templates/fsr_logo_cropped}}



% Custom Bindings
% \newcommand{\codeline}[1]{
%	\alert{\texttt{#1}}
%}


% Presentation title
% TODO Change the topic of the lesson
\title{Dekoratoren und Exceptions - Übung}
\date{20. Januar 2022}
\begin{document}

\maketitle

\begin{frame}{Gliederung}
	\setbeamertemplate{section in toc}[sections numbered]
	\tableofcontents
\end{frame}


\section{Übung 1 - Referenzen}
\begin{frame}{Übung 1 - Referenzen}
	\lstinputlisting[firstline=0, lastline=2]{../../exercise/exercise_10/ref.py}
	 \only<1>{\alert{Was wird ausgeprintet ?}}\only<2>{\alert{8}}
\end{frame}

\begin{frame}{Übung 1 - Referenzen}
	\lstinputlisting[firstline=4, lastline=20]{../../exercise/exercise_10/ref.py}
	\only<1>{\alert{Was wird ausgeprintet ?}}\only<2>{\alert{1 1 2 3 5}}
\end{frame}

\section{Übung 2 - Dekoratoren / Exceptions}
\begin{frame}{Übung 2 - Dekoratoren}
	Schreibe eine Funktion die andere Funktionen dekorieren kann. In der Original Funktion auftretende Exceptions sollen von dem Dekorator abgefangen werden.	
\end{frame}

\begin{frame}{Übung 2 - Dekoratoren}
	Schreibe eine Funktion \alert{f}, die als Parameter einen Fehlertyp und eine Reaktion auf den Fehler (Funktion) entgegen nimmt. f soll eine Funktion zurück geben, die als Dekorator verwendet werden kann. Der zurückgegebene Dekorator soll die Exception abfangen, falls sie in f angegeben wurde.
	\linebreak\linebreak
	Beispiel Anwendung:
	\lstinputlisting[firstline=16, lastline=18]{../../exercise/exercise_10/dec.py}
\end{frame}

\end{document}
