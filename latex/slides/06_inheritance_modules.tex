% The Slide Definitions
\input{../templates/course_definitions}

% Author and Course information
% This Document contains the information about this course.

% Authors of the slides
\author{Richard Müller, Tom Felber}

% Name of the Course
\institute{Python-Kurs}

% Fancy Logo 
\titlegraphic{\hfill\includegraphics[height=1.25cm]{../templates/fsr_logo_cropped}}



% Custom Bindings
% \newcommand{\codeline}[1]{
%	\alert{\texttt{#1}}
%}


% Presentation title
\title{Vererbung und Module}
\date{25. November 2021}

\begin{document}
	
\maketitle

\begin{frame}{Gliederung}
	\setbeamertemplate{section in toc}[sections numbered]
	\tableofcontents
\end{frame}

\section{Wiederholung}
\begin{frame}{Wiederholung}
	
\end{frame}

\section{Gesamtübersicht}
\begin{frame}{Gesamtübersicht}
	\textbf{Themen der nächsten Stunden}
	\begin{itemize}
		\item Referenzen Erklärung
		\item  \alert{Klassen}
		\item \alert{Imports}
		\item Nützliche Funktionen zur Iteration
		\item Lambda
		\item File handeling
		\item Listcomprehension
		\item Unpacking
		\item Dekoratoren
	\end{itemize}
\end{frame}

\section{Vererbung}
\begin{frame}{Vererbung}
	
\end{frame}

\section{Module - Einführung}
\begin{frame}{Module}
	Module sind Dateien, die Python-Ausdrücke und Definitionen enthalten. 
	
	Sie enden immer mit \alert{.py}. Diese Dateien kennen wir schon, nämlich als normale Scripte. Das Besondere ist jedoch, dass man diese Module als Objekt in sein eigentliches Script einbinden und nutzen kann.
	
\end{frame}

\begin{frame}{Import}
	Über das \codeline{import}-Statement kann man Module in sein Script einfügen. Es gibt zwei Möglichkeiten, dieses Statement zu verwenden.
	
	 Im Folgenden gehen wir davon aus, dass sich das ausgeführte Script im selben Verzeichnis wie die Datei \texttt{mein\_modul.py} befindet. In der Datei \texttt{mein\_modul.py} soll es eine Klasse namens \texttt{ImportiereMich} geben.
	 \pause
	
	\textbf{Möglichkeit 1:}
		\lstinputlisting[firstline=0,lastline=2]{resources/06_inheritance_modules/modules.py}
		\pause
	\textbf{Möglichkeit 2:}
		\lstinputlisting[firstline=4,lastline=5]{resources/06_inheritance_modules/modules.py}
	
\end{frame}

\begin{frame}{Import}
	Nutzt man bei diesem Beispiel Möglichkeit 1, so verhält sich das Modul wie ein normales Objekt.
	\lstinputlisting[firstline=0,lastline=2]{resources/06_inheritance_modules/modules.py}
	Mit dem \codeline{.} kann man auf Attribute und Methoden zugreifen, was hier nun Funktionen, Klassen und Variablen sind.
	
\end{frame}
\end{document}
