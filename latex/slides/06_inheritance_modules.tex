% The Slide Definitions
%document
\documentclass[10pt]{beamer}
%theme
\usetheme{metropolis}
% packages
\usepackage{color}
\usepackage{listings}
\usepackage[ngerman]{babel}
\usepackage[utf8]{inputenc}
\usepackage{multicol}


% color definitions
\definecolor{mygreen}{rgb}{0,0.6,0}
\definecolor{mygray}{rgb}{0.5,0.5,0.5}
\definecolor{mymauve}{rgb}{0.58,0,0.82}

\lstset{
    backgroundcolor=\color{white},
    % choose the background color;
    % you must add \usepackage{color} or \usepackage{xcolor}
    basicstyle=\footnotesize\ttfamily,
    % the size of the fonts that are used for the code
    breakatwhitespace=false,
    % sets if automatic breaks should only happen at whitespace
    breaklines=true,                 % sets automatic line breaking
    captionpos=b,                    % sets the caption-position to bottom
    commentstyle=\color{mygreen},    % comment style
    % deletekeywords={...},
    % if you want to delete keywords from the given language
    extendedchars=true,
    % lets you use non-ASCII characters;
    % for 8-bits encodings only, does not work with UTF-8
    literate={ä}{{\"a}}1 {ü}{{\"u}}1 {ö}{{\"o}}1 {Ä}{{\"A}}1 {Ü}{{\"U}}1 {Ö}{{\"O}}1 {ß}{{\ss{}}}1,
    % escapes umlauts
    frame=single,                    % adds a frame around the code
    keepspaces=true,
    % keeps spaces in text,
    % useful for keeping indentation of code
    % (possibly needs columns=flexible)
    keywordstyle=\color{blue},       % keyword style
    % morekeywords={*,...},
    % if you want to add more keywords to the set
    numbers=left,
    % where to put the line-numbers; possible values are (none, left, right)
    numbersep=5pt,
    % how far the line-numbers are from the code
    numberstyle=\tiny\color{mygray},
    % the style that is used for the line-numbers
    rulecolor=\color{black},
    % if not set, the frame-color may be changed on line-breaks
    % within not-black text (e.g. comments (green here))
    stepnumber=1,
    % the step between two line-numbers.
    % If it's 1, each line will be numbered
    stringstyle=\color{mymauve},     % string literal style
    tabsize=4,                       % sets default tabsize to 4 spaces
    % show the filename of files included with \lstinputlisting;
    % also try caption instead of title
    language = Python,
	showspaces = false,
	showtabs = false,
	showstringspaces = false,
	escapechar = ,
}

\def\ContinueLineNumber{\lstset{firstnumber=last}}
\def\StartLineAt#1{\lstset{firstnumber=#1}}
\let\numberLineAt\StartLineAt



\newcommand{\codeline}[1]{
	\alert{\texttt{#1}}
}


% Author and Course information
% This Document contains the information about this course.

% Authors of the slides
\author{Richard Müller, Tom Felber}

% Name of the Course
\institute{Python-Kurs}

% Fancy Logo 
\titlegraphic{\hfill\includegraphics[height=1.25cm]{../templates/fsr_logo_cropped}}



% Custom Bindings
% \newcommand{\codeline}[1]{
%	\alert{\texttt{#1}}
%}


% Presentation title
\title{Vererbung und Module}
\date{25. November 2021}

\begin{document}
	
\maketitle

\begin{frame}{Gliederung}
	\setbeamertemplate{section in toc}[sections numbered]
	\tableofcontents
\end{frame}

\section{Wiederholung}
\begin{frame}{Wiederholung}
	
\end{frame}

\section{Gesamtübersicht}
\begin{frame}{Gesamtübersicht}
	\textbf{Themen der nächsten Stunden}
	\begin{itemize}
		\item Referenzen Erklärung
		\item  \alert{Klassen}
		\item \alert{Imports}
		\item Nützliche Funktionen zur Iteration
		\item Lambda
		\item File handeling
		\item Listcomprehension
		\item Unpacking
		\item Dekoratoren
	\end{itemize}
\end{frame}

\section{Vererbung}
\begin{frame}{Vererbung}
	
\end{frame}

\section{Module - Einführung}
\begin{frame}{Module}
	Module sind Dateien, die Python-Ausdrücke und Definitionen enthalten. 
	
	Sie enden immer mit \alert{.py}. Diese Dateien kennen wir schon, nämlich als normale Scripte. Das Besondere ist jedoch, dass man diese Module als Objekt in sein eigentliches Script einbinden und nutzen kann.
	
\end{frame}

\begin{frame}{Import}
	Über das \codeline{import}-Statement kann man Module in sein Script einfügen. Es gibt zwei Möglichkeiten, dieses Statement zu verwenden.
	
	 Im Folgenden gehen wir davon aus, dass sich das ausgeführte Script im selben Verzeichnis wie die Datei \texttt{mein\_modul.py} befindet. In der Datei \texttt{mein\_modul.py} soll es eine Klasse namens \texttt{ImportiereMich} geben.
	 \pause
	
	\textbf{Möglichkeit 1:}
		\lstinputlisting[firstline=0,lastline=2]{resources/06_inheritance_modules/modules.py}
		\pause
	\textbf{Möglichkeit 2:}
		\lstinputlisting[firstline=4,lastline=5]{resources/06_inheritance_modules/modules.py}
	
\end{frame}

\begin{frame}{Import}
	Nutzt man bei diesem Beispiel Möglichkeit 1, so verhält sich das Modul wie ein normales Objekt.
	\lstinputlisting[firstline=0,lastline=2]{resources/06_inheritance_modules/modules.py}
	Mit dem \codeline{.} kann man auf Attribute und Methoden zugreifen, was hier nun Funktionen, Klassen und Variablen sind.
	
\end{frame}
\end{document}
