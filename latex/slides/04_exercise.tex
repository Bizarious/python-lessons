% The Slide Definitions
\input{../templates/course_definitions}

% Author and Course information
% This Document contains the information about this course.

% Authors of the slides
\author{Richard Müller, Tom Felber}

% Name of the Course
\institute{Python-Kurs}

% Fancy Logo 
\titlegraphic{\hfill\includegraphics[height=1.25cm]{../templates/fsr_logo_cropped}}



% Custom Bindings
% \newcommand{\codeline}[1]{
%	\alert{\texttt{#1}}
%}


% Presentation title
% TODO Change the topic of the lesson
\title{Tupel und Dictionaries - Übung}
\date{11. November 2021}
\begin{document}

\maketitle

\begin{frame}{Gliederung}
	\setbeamertemplate{section in toc}[sections numbered]
	\tableofcontents
\end{frame}

% TODO: Add your content right below here.

\section{Tupel}

\begin{frame}{Übung 1}
	\begin{itemize}
		\item[\textbf{1.}] Schreibe eine Funktion, die zwei 2-dimensionale Punkte A und B (jeweils als Tupel) entgegennimmt und den Vektor AB als Tupel zurückgibt.
		\begin{itemize}
			\item[\textbf{Bsp.:}] A = (3, 5), B = (-1, -2), Ergebnis: (-4, -7)
		\end{itemize}
		\item[\textbf{2.}] Modifiziere die Funktion so, dass sie nun Vektoren beliebiger Dimension berechnen kann. Der Einfachheit halber gehen wir davon aus, dass beide Punkte die selbe Dimension haben.
	\end{itemize}
\end{frame}

\end{document}
