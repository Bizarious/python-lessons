% The Slide Definitions
%document
\documentclass[10pt]{beamer}
%theme
\usetheme{metropolis}
% packages
\usepackage{color}
\usepackage{listings}
\usepackage[ngerman]{babel}
\usepackage[utf8]{inputenc}
\usepackage{multicol}


% color definitions
\definecolor{mygreen}{rgb}{0,0.6,0}
\definecolor{mygray}{rgb}{0.5,0.5,0.5}
\definecolor{mymauve}{rgb}{0.58,0,0.82}

\lstset{
    backgroundcolor=\color{white},
    % choose the background color;
    % you must add \usepackage{color} or \usepackage{xcolor}
    basicstyle=\footnotesize\ttfamily,
    % the size of the fonts that are used for the code
    breakatwhitespace=false,
    % sets if automatic breaks should only happen at whitespace
    breaklines=true,                 % sets automatic line breaking
    captionpos=b,                    % sets the caption-position to bottom
    commentstyle=\color{mygreen},    % comment style
    % deletekeywords={...},
    % if you want to delete keywords from the given language
    extendedchars=true,
    % lets you use non-ASCII characters;
    % for 8-bits encodings only, does not work with UTF-8
    literate={ä}{{\"a}}1 {ü}{{\"u}}1 {ö}{{\"o}}1 {Ä}{{\"A}}1 {Ü}{{\"U}}1 {Ö}{{\"O}}1 {ß}{{\ss{}}}1,
    % escapes umlauts
    frame=single,                    % adds a frame around the code
    keepspaces=true,
    % keeps spaces in text,
    % useful for keeping indentation of code
    % (possibly needs columns=flexible)
    keywordstyle=\color{blue},       % keyword style
    % morekeywords={*,...},
    % if you want to add more keywords to the set
    numbers=left,
    % where to put the line-numbers; possible values are (none, left, right)
    numbersep=5pt,
    % how far the line-numbers are from the code
    numberstyle=\tiny\color{mygray},
    % the style that is used for the line-numbers
    rulecolor=\color{black},
    % if not set, the frame-color may be changed on line-breaks
    % within not-black text (e.g. comments (green here))
    stepnumber=1,
    % the step between two line-numbers.
    % If it's 1, each line will be numbered
    stringstyle=\color{mymauve},     % string literal style
    tabsize=4,                       % sets default tabsize to 4 spaces
    % show the filename of files included with \lstinputlisting;
    % also try caption instead of title
    language = Python,
	showspaces = false,
	showtabs = false,
	showstringspaces = false,
	escapechar = ,
}

\def\ContinueLineNumber{\lstset{firstnumber=last}}
\def\StartLineAt#1{\lstset{firstnumber=#1}}
\let\numberLineAt\StartLineAt



\newcommand{\codeline}[1]{
	\alert{\texttt{#1}}
}


% Author and Course information
% This Document contains the information about this course.

% Authors of the slides
\author{Richard Müller, Tom Felber}

% Name of the Course
\institute{Python-Kurs}

% Fancy Logo 
\titlegraphic{\hfill\includegraphics[height=1.25cm]{../templates/fsr_logo_cropped}}



% Custom Bindings
% \newcommand{\codeline}[1]{
%	\alert{\texttt{#1}}
%}


% Presentation title
% TODO Change the topic of the lesson
\title{Tupel und Dictionaries - Übung}
\date{11. November 2021}
\begin{document}

\maketitle

\begin{frame}{Gliederung}
	\setbeamertemplate{section in toc}[sections numbered]
	\tableofcontents
\end{frame}

% TODO: Add your content right below here.

\section{Tupel}

\begin{frame}{Übung 1}
	\begin{itemize}
		\item[\textbf{1.}] Schreibe eine Funktion, die zwei 2-dimensionale Punkte A und B (jeweils als Tupel) entgegennimmt und den Vektor AB als Tupel zurückgibt.
		\begin{itemize}
			\item[\textbf{Bsp.:}]\codeline{A = (3, 5), B = (-1, -2), Ergebnis: (-4, -7)}
		\end{itemize}
		\item[\textbf{2.}] Modifiziere die Funktion so, dass sie nun Vektoren beliebiger Dimension berechnen kann. Der Einfachheit halber gehen wir davon aus, dass beide Punkte die selbe Dimension haben.
	\end{itemize}
\end{frame}

\begin{frame}{Übung 2}
	\begin{itemize}
		\item[\textbf{1.}] Es ist eine Liste von Variablen verschiedener Typen gegeben. Schreibe eine Funktion, die ein dictionary zurück gibt, in dem die Variablen als Keys verwendet werden und ihr jeweiliger Typ als Value.
		\begin{itemize}
			\item[\textbf{Bsp.:}] \codeline{[1, 'hallo'] Ergebnis: \{1: 'int', 'hallo': 'str'\}}
		\end{itemize}
		\item[\textbf{2.}] Definiere eine zweite Funktion, die stattdessen den Typ einer Variable als Key verwendet. Alle Variablen eines Typs sollen, in einer Liste als Value für den jeweiligen Key, abgespeichert werden.
		\begin{itemize}
			\item[\textbf{Bsp.:}]\codeline{[1, 2, 4, 'hallo', 'ok'] Ergebnis: \{'int': [1, 2, 4], 'str': ['hallo', 'ok']\}}
		\end{itemize}
		\end{itemize}
\end{frame}

\begin{frame}{Übung 3}
	\begin{itemize}
		\item[\textbf{1.}]Gegeben ist ein Dictionary. Schreibe eine Funktion, welche zählt, wie oft jeder Value vorkommt und ein Dictionary ausgibt, indem der jeweilige Value auf die Anzahl seines Vorkommens gemappt wird.
		\begin{itemize}
			\item[\textbf{Bsp.:}]\codeline{\{'hallo': 'eins', 'x': 'eins', 'hi': 'ok'\} Ergebnis: \{'eins': 2, 'ok': 1\}}
		\end{itemize}
		\item[\textbf{2.}] Benutze die in \textbf{1.} definierte Funktion, um von einem gegebenen Dictionary der Value zu bestimmen, die am häufigsten vorkommt.
		\begin{itemize}
			\item[\textbf{Bsp.:}]\codeline{\{'eins': 2, 'ok': 1\} Ergebnis: 'eins'}
		\end{itemize}
	
		\pause
		\item[\textbf{Z}] Schreibe eine Funktion, die eine Liste von 2-dimensionalen Integer Tupeln entgegen nimmt. Die Tupelelemente sollen addiert werden und in einer Ergebnisliste ausgegeben werden. Da es vorkommen kann, dass das selbe Tupel in der Liste zweimal vorkommt, soll ein Dictionary benutzt werden, um berechnete Werte zu speichern, sodass diese nicht erneut berechnet werden müssen.
		\begin{itemize}
			\item[\textbf{Bsp.:}]\codeline{[(65, 7), (4, 5), (65, 7)] Ergebnis: [3, 72, 9, 72]}
		\end{itemize}
	\end{itemize}
\end{frame}


\end{document}
