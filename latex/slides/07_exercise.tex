% The Slide Definitions
%document
\documentclass[10pt]{beamer}
%theme
\usetheme{metropolis}
% packages
\usepackage{color}
\usepackage{listings}
\usepackage[ngerman]{babel}
\usepackage[utf8]{inputenc}
\usepackage{multicol}


% color definitions
\definecolor{mygreen}{rgb}{0,0.6,0}
\definecolor{mygray}{rgb}{0.5,0.5,0.5}
\definecolor{mymauve}{rgb}{0.58,0,0.82}

\lstset{
    backgroundcolor=\color{white},
    % choose the background color;
    % you must add \usepackage{color} or \usepackage{xcolor}
    basicstyle=\footnotesize\ttfamily,
    % the size of the fonts that are used for the code
    breakatwhitespace=false,
    % sets if automatic breaks should only happen at whitespace
    breaklines=true,                 % sets automatic line breaking
    captionpos=b,                    % sets the caption-position to bottom
    commentstyle=\color{mygreen},    % comment style
    % deletekeywords={...},
    % if you want to delete keywords from the given language
    extendedchars=true,
    % lets you use non-ASCII characters;
    % for 8-bits encodings only, does not work with UTF-8
    literate={ä}{{\"a}}1 {ü}{{\"u}}1 {ö}{{\"o}}1 {Ä}{{\"A}}1 {Ü}{{\"U}}1 {Ö}{{\"O}}1 {ß}{{\ss{}}}1,
    % escapes umlauts
    frame=single,                    % adds a frame around the code
    keepspaces=true,
    % keeps spaces in text,
    % useful for keeping indentation of code
    % (possibly needs columns=flexible)
    keywordstyle=\color{blue},       % keyword style
    % morekeywords={*,...},
    % if you want to add more keywords to the set
    numbers=left,
    % where to put the line-numbers; possible values are (none, left, right)
    numbersep=5pt,
    % how far the line-numbers are from the code
    numberstyle=\tiny\color{mygray},
    % the style that is used for the line-numbers
    rulecolor=\color{black},
    % if not set, the frame-color may be changed on line-breaks
    % within not-black text (e.g. comments (green here))
    stepnumber=1,
    % the step between two line-numbers.
    % If it's 1, each line will be numbered
    stringstyle=\color{mymauve},     % string literal style
    tabsize=4,                       % sets default tabsize to 4 spaces
    % show the filename of files included with \lstinputlisting;
    % also try caption instead of title
    language = Python,
	showspaces = false,
	showtabs = false,
	showstringspaces = false,
	escapechar = ,
}

\def\ContinueLineNumber{\lstset{firstnumber=last}}
\def\StartLineAt#1{\lstset{firstnumber=#1}}
\let\numberLineAt\StartLineAt



\newcommand{\codeline}[1]{
	\alert{\texttt{#1}}
}


% Author and Course information
% This Document contains the information about this course.

% Authors of the slides
\author{Richard Müller, Tom Felber}

% Name of the Course
\institute{Python-Kurs}

% Fancy Logo 
\titlegraphic{\hfill\includegraphics[height=1.25cm]{../templates/fsr_logo_cropped}}



% Custom Bindings
% \newcommand{\codeline}[1]{
%	\alert{\texttt{#1}}
%}


% Presentation title
% TODO Change the topic of the lesson
\title{Module - Übung}
\date{2. November 2021}
\begin{document}

\maketitle

\begin{frame}{Gliederung}
	\setbeamertemplate{section in toc}[sections numbered]
	\tableofcontents
\end{frame}

\section{Übung 1 - Referenzen}
\begin{frame}{Referenzen - Übung 1.1}
	\large\lstinputlisting[firstline=0, lastline=8]{../../exercise/exercise_07/ref.py}
	b = \only<1>{?}\only<2>{\alert{2}}
\end{frame}
\begin{frame}{Referenzen - Übung 1.2}
	\large\lstinputlisting[firstline=10, lastline=12]{../../exercise/exercise_07/ref.py}
	a = \only<1>{?}\only<2>{\alert{[1, 2, [1, 2, [1, 2, ...]]]}}
	
	b = \only<1>{?}\only<2>{\alert{1}}
\end{frame}

\section{Übung 2 - Module}
\begin{frame}{Time - Übung 2.1}
	\begin{itemize}
		\item[\textbf{a)}]Definiere eine Klasse \codeline{Timer} mit zwei Funktionen: \codeline{execute} und \codeline{action}. Die \codeline{action} Funktion soll leer bleiben (pass). In der \codeline{execute} Funktion soll die \codeline{action} Funktion ausgeführt werden und die Zeit der Ausführung gestoppt und ausgegeben werden (\codeline{time.time()} benutzen).
		\linebreak
		
		\item[\textbf{b)}]Erstelle eine neue Klasse und überschreibe die \codeline{action} Funktion mit einer Funktion, die das Programm 3 Sekunden lang 'schlafen legt'. Lass dir das Ergebnis der \codeline{execute} Funktion deiner neuen Klasse ausgeben. 
		\linebreak
		\begin{itemize}
			\item[\textbf{Hinweis:}] Lass dir das Ergebnis von \codeline{execute} in einer \codeline{\_\_name\_\_ == '\_\_main\_\_'} Abfrage ausgeben.
		\end{itemize}
	\end{itemize}
\end{frame}

\begin{frame}{Itertools}
	Importiere \codeline{itertools}
	\linebreak
	Erstelle eine neue Klasse die von \codeline{Timer} erbt und in der die\codeline{action} Methode alle \only<1>{3}\only<2>{\alert{5}}-stelligen Kombinationen der Zahlen von 0 bis 19 als Liste von Tupeln ausgibt. 
	\begin{itemize}
		\item[\textbf{a)}] löse die Aufgabe mit verschachtelten \codeline{for} Schleifen
		\item[\textbf{b)}] löse die Aufgabe mit der \codeline{itertool.product} Funktion
	\end{itemize}
	Vergleiche die beiden Ausführungszeiten.
\end{frame}

\end{document}
