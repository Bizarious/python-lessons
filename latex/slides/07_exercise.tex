% The Slide Definitions
\input{../templates/course_definitions}

% Author and Course information
% This Document contains the information about this course.

% Authors of the slides
\author{Richard Müller, Tom Felber}

% Name of the Course
\institute{Python-Kurs}

% Fancy Logo 
\titlegraphic{\hfill\includegraphics[height=1.25cm]{../templates/fsr_logo_cropped}}



% Custom Bindings
% \newcommand{\codeline}[1]{
%	\alert{\texttt{#1}}
%}


% Presentation title
% TODO Change the topic of the lesson
\title{Module - Übung}
\date{2. November 2021}
\begin{document}

\maketitle

\begin{frame}{Gliederung}
	\setbeamertemplate{section in toc}[sections numbered]
	\tableofcontents
\end{frame}

\section{Übung 1 - Referenzen}
\begin{frame}{Referenzen - Übung 1.1}
	\large\lstinputlisting[firstline=0, lastline=8]{../../exercise/exercise_07/ref.py}
	b = \only<1>{?}\only<2>{\alert{2}}
\end{frame}
\begin{frame}{Referenzen - Übung 1.2}
	\large\lstinputlisting[firstline=10, lastline=12]{../../exercise/exercise_07/ref.py}
	a = \only<1>{?}\only<2>{\alert{[1, 2, [1, 2, [1, 2, ...]]]}}
	
	b = \only<1>{?}\only<2>{\alert{1}}
\end{frame}

\section{Übung 2 - Module}
\begin{frame}{Time - Übung 2.1}
	\begin{itemize}
		\item[\textbf{a)}]Definiere eine Klasse \codeline{Timer} mit zwei Funktionen: \codeline{execute} und \codeline{action}. Die \codeline{action} Funktion soll leer bleiben (pass). In der \codeline{execute} Funktion soll die \codeline{action} Funktion ausgeführt werden und die Zeit der Ausführung gestoppt und ausgegeben werden (\codeline{time.time()} benutzen).
		\linebreak
		
		\item[\textbf{b)}]Erstelle eine neue Klasse und überschreibe die \codeline{action} Funktion mit einer Funktion, die das Programm 3 Sekunden lang 'schlafen legt'. Lass dir das Ergebnis der \codeline{execute} Funktion deiner neuen Klasse ausgeben. 
		\linebreak
		\begin{itemize}
			\item[\textbf{Hinweis:}] Lass dir das Ergebnis von \codeline{execute} in einer \codeline{\_\_name\_\_ == \_\_main\_\_} Abfrage ausgeben.
		\end{itemize}
	\end{itemize}
\end{frame}

\begin{frame}{Itertools}
	Importiere \codeline{itertools}
	\linebreak
	Erstelle eine neue Klasse die von \codeline{Timer} erbt und in der die\codeline{action} Methode alle \only<1>{3}\only<2>{\alert{5}}-stelligen Kombinationen der Zahlen von 0 bis 19 als Liste von Tupeln ausgibt. 
	\begin{itemize}
		\item[\textbf{a)}] löse die Aufgabe mit verschachtelten \codeline{for} Schleifen
		\item[\textbf{b)}] löse die Aufgabe mit der \codeline{itertool.product} Funktion
	\end{itemize}
	Vergleiche die beiden Ausführungszeiten.

	
	 
\end{frame}

\end{document}
