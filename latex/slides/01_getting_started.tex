% The Slide Definitions
%document
\documentclass[10pt]{beamer}
%theme
\usetheme{metropolis}
% packages
\usepackage{color}
\usepackage{listings}
\usepackage[ngerman]{babel}
\usepackage[utf8]{inputenc}
\usepackage{multicol}


% color definitions
\definecolor{mygreen}{rgb}{0,0.6,0}
\definecolor{mygray}{rgb}{0.5,0.5,0.5}
\definecolor{mymauve}{rgb}{0.58,0,0.82}

\lstset{
    backgroundcolor=\color{white},
    % choose the background color;
    % you must add \usepackage{color} or \usepackage{xcolor}
    basicstyle=\footnotesize\ttfamily,
    % the size of the fonts that are used for the code
    breakatwhitespace=false,
    % sets if automatic breaks should only happen at whitespace
    breaklines=true,                 % sets automatic line breaking
    captionpos=b,                    % sets the caption-position to bottom
    commentstyle=\color{mygreen},    % comment style
    % deletekeywords={...},
    % if you want to delete keywords from the given language
    extendedchars=true,
    % lets you use non-ASCII characters;
    % for 8-bits encodings only, does not work with UTF-8
    literate={ä}{{\"a}}1 {ü}{{\"u}}1 {ö}{{\"o}}1 {Ä}{{\"A}}1 {Ü}{{\"U}}1 {Ö}{{\"O}}1 {ß}{{\ss{}}}1,
    % escapes umlauts
    frame=single,                    % adds a frame around the code
    keepspaces=true,
    % keeps spaces in text,
    % useful for keeping indentation of code
    % (possibly needs columns=flexible)
    keywordstyle=\color{blue},       % keyword style
    % morekeywords={*,...},
    % if you want to add more keywords to the set
    numbers=left,
    % where to put the line-numbers; possible values are (none, left, right)
    numbersep=5pt,
    % how far the line-numbers are from the code
    numberstyle=\tiny\color{mygray},
    % the style that is used for the line-numbers
    rulecolor=\color{black},
    % if not set, the frame-color may be changed on line-breaks
    % within not-black text (e.g. comments (green here))
    stepnumber=1,
    % the step between two line-numbers.
    % If it's 1, each line will be numbered
    stringstyle=\color{mymauve},     % string literal style
    tabsize=4,                       % sets default tabsize to 4 spaces
    % show the filename of files included with \lstinputlisting;
    % also try caption instead of title
    language = Python,
	showspaces = false,
	showtabs = false,
	showstringspaces = false,
	escapechar = ,
}

\def\ContinueLineNumber{\lstset{firstnumber=last}}
\def\StartLineAt#1{\lstset{firstnumber=#1}}
\let\numberLineAt\StartLineAt



\newcommand{\codeline}[1]{
	\alert{\texttt{#1}}
}


% Author and Course information
% This Document contains the information about this course.

% Authors of the slides
\author{Richard Müller, Tom Felber}

% Name of the Course
\institute{Python-Kurs}

% Fancy Logo 
\titlegraphic{\hfill\includegraphics[height=1.25cm]{../templates/fsr_logo_cropped}}



% Custom Bindings
% \newcommand{\codeline}[1]{
%	\alert{\texttt{#1}}
%}


% Presentation title
\title{Grundlagen}
\date{\today}


\begin{document}

\maketitle

\begin{frame}{Gliederung}
	\begin{multicols}{2}
		\setbeamertemplate{section in toc}[sections numbered]
		\tableofcontents
	\end{multicols}
\end{frame}

% --------------------------- Über den Kurs -----------------------------------
\section{Über diesen Kurs}
\begin{frame}{Über diesen Kurs}
	\begin{itemize}
    	\item 12 Kurseinheiten
    	\item setzt grundlegende Programmierkenntnisse voraus
    	\item Ressourcen
    	\begin{itemize}
    	    \item \href{https://kurse.ifsr.de/}{die Kursseite} %TODO: add link to our group once it exists
    	    \item \href{docs.python.org}{offizielle Dokumentation}
    	    \item die \href{https://github.com/Bizarious/python-course}{Github-Seite} für diesen Kurs
    	\end{itemize}
    	\item Hinweis: SCM's sind hilfreich (\href{https://git-scm.com}{git})
	\end{itemize}
\end{frame}


% ----------------------- Der Python Interpreter ------------------------------
\section{Der Python Interpreter}
\begin{frame}{Der Python Interpreter}
	\begin{itemize}
    	\item Die zwei verbreitet verwendeten Python Versionen sind 2.7 und 3, wir werden 3 benutzen, da 2.7 nicht mehr unterstützt wird
		\item Python kann \alert{\href{http://www.python.org}{hier}} heruntergeladen und installiert werden oder mit dem Paketmanager eurer Wahl. (Das Paket sollte \texttt{python3} und \texttt{python3-dev} sein, außer unter Arch)
    	\item Python funktioniert am besten unter UNIX (ist aber okay unter Windows)
    	\item Den Interpreter startet man mit \texttt{python3} im Terminal oder mit \texttt{Python.exe}
    	\item Der Interpreter stellt die volle Funktionalität von Python bereit, einschließlich dem Erstellen von Klassen und Funktionen
	\end{itemize}
\end{frame}

% --------------------------- Python Scripte ----------------------------------
\section{Python Scripte}
\begin{frame}{Python Scripte}
\begin{description}
   	\item[Editor] empfohlen
    \begin{itemize}
        \item \href{https://atom.io}{atom} (benutzen wir im Kurs)
    \end{itemize}
    \item[IDEs] hilfreich bei grö\ss{}eren Projekten, Vorstellung gegen Ende des Kurses oder auf Anfrage
    \begin{itemize}
       	\item \href{https://jetbrains.com/pycharm}{PyCharm} (free + professional für Studenten)
   	\end{itemize}
\end{description}
\end{frame}
\begin{frame}{Python Scripte}
\begin{description}
   	\item[Struktur]
   	\begin{itemize}
       	\item Python Scripte sind Textdateien, die auf \texttt{.py} enden
        \item Python Packages sind Ordner mit einer \texttt{\_\_init\_\_.py} Datei (behandeln wir später)
    \end{itemize}
\end{description}
\end{frame}


% ----------------------- Grundlagen der Sprache ------------------------------
\section{Grundlagen der Sprache}
\begin{frame}[fragile]{Grundlagen der Sprache}
    Python ist eine schwach typisierte Scriptsprache (weakly typed scripting language). Es gibt Typen (anders als in JavaScript), aber Variablen haben keine festen Typen.\\

    \textbf{Beispiel (erzeugt keinen Fehler):}
    \lstinputlisting{resources/01_getting_started/types.py}
\end{frame}

\begin{frame}
	\textbf{builtin Datentypen:}\\
	\begin{tabular}{c|l}
		Name & Funktion \\ \hline
		\texttt{object} & Basistyp, alles erbt von \texttt{object} \\
		\texttt{int} & Ganzzahl "beliebiger" Größe \\
		\texttt{float} & Kommazahl "beliebiger" Größe \\
		\texttt{bool} & Wahrheitswert (\texttt{True}, \texttt{False})\\
		\texttt{None} & Typ des \texttt{None}-Objektes \\
		\texttt{str} & Zeichenkette \\
		\texttt{type} & Grundtyp aller Typen (z.B. \texttt{int} ist eine Instanz von \texttt{int}) \\
		\texttt{list} & standard Liste \\
		\texttt{tuple} & unveränderbares n-Tupel \\
		\texttt{set} & (mathematische) Menge von Objekten \\
		\texttt{frozenset} & unveränderbare (mathematische) Menge von Objekten \\
		\texttt{dict} & Hash-Map \\
	\end{tabular}
\end{frame}


% ------------------------- Das erste Programm --------------------------------
\section{Das erste Programm}
\begin{frame}[fragile]{Das erste Programm}
	Ein simples 'Hallo Welt'-Programm:\\[.5cm]
	\lstinputlisting{resources/01_getting_started/basic_hello_world.py}
	gute Konvention:\\[.5cm]
	\lstinputlisting{resources/01_getting_started/hello_world.py}
\end{frame}


% ----------------------- Wichtige Eigenschaften ------------------------------
\begin{frame}[fragile]{Das erste Programm}
	\textbf{Wichtige Eigenschaften:}
	\begin{itemize}
	    \item Keine Semikolons
	    \item Keine geschweiften Klammern für Codeblöcke
	    \item Einrückungen zeigen Codeblöcke an
	    \item Funktionsaufrufe immer mit runden Klammern
	    \item Funktionen definieren mit \\
	   		  \alert{\texttt{def <funktionsname>([parameter\_liste, ...]):}}
	    \item Variablen mit der Struktur \alert{\texttt{\_\_name\_\_}} sind spezielle Werte (gewöhnlich aus \alert{\texttt{builtin}} oder Methoden von Standardtypen)
	\end{itemize}
\end{frame}


% ------------------------------ Operatoren -----------------------------------
\section{Operatoren}
\begin{frame}[fragile]{Operatoren}
	\begin{description}
	    \item[mathematisch] \alert{\texttt{+}}, \alert{\texttt{-}}, \alert{\texttt{*}}, \alert{\texttt{/}}, \alert{\texttt{\%}}
	    \item[vergleichend] \alert{\texttt{<}}, \alert{\texttt{>}}, \alert{\texttt{<=}}, \alert{\texttt{>=}}, \alert{\texttt{==}} (Wert gleich), \alert{\texttt{is}} (gleiches Objekt/gleiche Referenz)
	    \item[logisch] \alert{\texttt{and}}, \alert{\texttt{or}}, \alert{\texttt{not}}\alert{\texttt{(a \&\& b) || (!c)}} aus C oder Java entspricht \alert{\texttt{(a and b) or not c}} in Python
	    \item[bitweise] \alert{\texttt{\&}}, \alert{\texttt{|}}, \alert{\texttt{<<}}, \alert{\texttt{>>}}, \alert{\texttt{\^}} (xor), \alert{\texttt{\~}} (invertieren)
	    \item[Accessoren] \alert{\texttt{.}} (für Methoden und Attribute), \alert{\texttt{[]}} (für Datenstrukturen mit Index)
	\end{description}
\end{frame}


% -------------------------- Namenskonvention -------------------------------
\section{Namenskonvention}
\begin{frame}[fragile]{Namenskonvention}
	\begin{description}
	    \item[\textbf{Klassen}] \textit{PascalCase}, alles direkt zusammen, groß beginnend und jedes neue Wort groß
	    \item[\textbf{Variablen, Funktionen, Methoden}] \textit{snake\_case}, alles klein und Wörter mit Unterstrich getrennt \\
	    \textbf{Merke:} Da \alert{\texttt{-}} ein Operator ist, ist es in Namen von Variablen, Funktionen etc. \textbf{nicht} zulässig (damit Python eine Kontextfreie Sprache ist)
	    \item[\textbf{protected Variablen, Funktionen, Methoden}] beginnen mit einem Unterstrich \alert{\texttt{\_}} oder mit zweien \alert{\texttt{\_\_}} für private
	    \item[\textbf{Merke}] Python hat kein Zugriffsmanagement. Die Regel mit dem Unterstrich ist nur eine Konvention um zu verhindern, dass andere Teile des Codes nutzen, der eine hohe Wahrscheinlichkeit hat in Zukunft verändert zu werden.
	\end{description}
\end{frame}


% ------------------------- Strings - Grundlagen -------------------------------
\section{Strings}
\subsection{Grundlagen}
\begin{frame}[fragile]{Strings - Grundlagen}
	\begin{itemize}
	    \item Der Typ eines Strings ist \alert{\texttt{str}}.
	    \item Strings sind in Python immutable (nicht veränderbar). Jede String Operation erzeugt einen neuen String.
	    \item Ein String kann erzeugt werden mit einer Zeichenkette in Anführungszeichen, \alert{\texttt{'{}'{}}} oder \alert{\texttt{"{}"{}}} (beide sind äquivalent).
	    \item rohe Strings mir dem Präfix \alert{\texttt{r}}, \alert{\texttt{r"mystring"}} oder \alert{\texttt{r'mystring'}}
	    \item Strings in Python3 sind UTF-8 encoded.
	\end{itemize}
\end{frame}


% ------------------------- Strings - Verknüpfen -------------------------------
\subsection{Verknüpfen}
\begin{frame}[fragile]{Strings - Verknüpfen}
	\begin{itemize}
	    \item Strings können durch Konkatenation verknüpft werden \\
	    \lstinputlisting[lastline=1]{resources/01_getting_started/string_concatenation.py}
	    \item Listen, Tupel etc. von Strings können via `str.join` verknüpft werden \\
	    \lstinputlisting[firstline=3]{resources/01_getting_started/string_concatenation.py}
	    Dabei ist der String, auf welchem die Methode aufgerufen wird, der Separator.
	\end{itemize}
\end{frame}


% ------------------------ Strings - Formatierung ------------------------------
\subsection{Formatierung}
\begin{frame}[fragile]{Strings - Formatierung}
	Wir wollen den String \alert{\texttt{'my string 4 vier'}} erzeugen.

	\lstinputlisting[lastline=10]{resources/01_getting_started/string_format.py}
\end{frame}

\begin{frame}[fragile]{Strings - Formatierung}
	Wir wollen den String \alert{\texttt{'my string 4 vier'}} erzeugen.
	\lstinputlisting[firstline=12, lastline=18]{resources/01_getting_started/string_format.py}
\end{frame}

\begin{frame}[fragile]{Strings - Formatierung}
		Wir wollen den String \alert{\texttt{'my string 4 vier'}} erzeugen.
	\lstinputlisting[firstline=20]{resources/01_getting_started/string_format.py}
\end{frame}


% nothing to do from here on
\end{document}
