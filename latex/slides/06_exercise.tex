% The Slide Definitions
%document
\documentclass[10pt]{beamer}
%theme
\usetheme{metropolis}
% packages
\usepackage{color}
\usepackage{listings}
\usepackage[ngerman]{babel}
\usepackage[utf8]{inputenc}
\usepackage{multicol}


% color definitions
\definecolor{mygreen}{rgb}{0,0.6,0}
\definecolor{mygray}{rgb}{0.5,0.5,0.5}
\definecolor{mymauve}{rgb}{0.58,0,0.82}

\lstset{
    backgroundcolor=\color{white},
    % choose the background color;
    % you must add \usepackage{color} or \usepackage{xcolor}
    basicstyle=\footnotesize\ttfamily,
    % the size of the fonts that are used for the code
    breakatwhitespace=false,
    % sets if automatic breaks should only happen at whitespace
    breaklines=true,                 % sets automatic line breaking
    captionpos=b,                    % sets the caption-position to bottom
    commentstyle=\color{mygreen},    % comment style
    % deletekeywords={...},
    % if you want to delete keywords from the given language
    extendedchars=true,
    % lets you use non-ASCII characters;
    % for 8-bits encodings only, does not work with UTF-8
    literate={ä}{{\"a}}1 {ü}{{\"u}}1 {ö}{{\"o}}1 {Ä}{{\"A}}1 {Ü}{{\"U}}1 {Ö}{{\"O}}1 {ß}{{\ss{}}}1,
    % escapes umlauts
    frame=single,                    % adds a frame around the code
    keepspaces=true,
    % keeps spaces in text,
    % useful for keeping indentation of code
    % (possibly needs columns=flexible)
    keywordstyle=\color{blue},       % keyword style
    % morekeywords={*,...},
    % if you want to add more keywords to the set
    numbers=left,
    % where to put the line-numbers; possible values are (none, left, right)
    numbersep=5pt,
    % how far the line-numbers are from the code
    numberstyle=\tiny\color{mygray},
    % the style that is used for the line-numbers
    rulecolor=\color{black},
    % if not set, the frame-color may be changed on line-breaks
    % within not-black text (e.g. comments (green here))
    stepnumber=1,
    % the step between two line-numbers.
    % If it's 1, each line will be numbered
    stringstyle=\color{mymauve},     % string literal style
    tabsize=4,                       % sets default tabsize to 4 spaces
    % show the filename of files included with \lstinputlisting;
    % also try caption instead of title
    language = Python,
	showspaces = false,
	showtabs = false,
	showstringspaces = false,
	escapechar = ,
}

\def\ContinueLineNumber{\lstset{firstnumber=last}}
\def\StartLineAt#1{\lstset{firstnumber=#1}}
\let\numberLineAt\StartLineAt



\newcommand{\codeline}[1]{
	\alert{\texttt{#1}}
}


% Author and Course information
% This Document contains the information about this course.

% Authors of the slides
\author{Richard Müller, Tom Felber}

% Name of the Course
\institute{Python-Kurs}

% Fancy Logo 
\titlegraphic{\hfill\includegraphics[height=1.25cm]{../templates/fsr_logo_cropped}}



% Custom Bindings
% \newcommand{\codeline}[1]{
%	\alert{\texttt{#1}}
%}


% Presentation title
% TODO Change the topic of the lesson
\title{Referenzen - Übung}
\date{18. November 2021}
\begin{document}

\maketitle

\begin{frame}{Gliederung}
	\setbeamertemplate{section in toc}[sections numbered]
	\tableofcontents
\end{frame}
% TODO: Add your content right below here.

\section{Übung 1}
\begin{frame}{Referenzen - Übung 1.1}
	\large\lstinputlisting[firstline=1, lastline=5]{../../exercise/exercise_06/ref.py}
	a = \only<1>{?}\only<2>{\alert{[ [1, 5], [1, 5] ]}}
	
	b = \only<1>{?}\only<2>{\alert{[1, 5]}}
\end{frame}


\begin{frame}{Referenzen - Übung 1.2}
	\large\lstinputlisting[firstline=8, lastline=12]{../../exercise/exercise_06/ref.py}
	a = \only<1>{?}\only<2>{\alert{[2, 3]}}
\end{frame}

\begin{frame}{Referenzen - Übung 1.3}
	\only<1>{
		\large\lstinputlisting[firstline=25, lastline=26]{../../exercise/exercise_06/ref.py}
		
		\large\lstinputlisting[firstline=20, lastline=21]{../../exercise/exercise_06/ref.py}
	}
	\only<2>{
		\large\lstinputlisting[firstline=15, lastline=21]{../../exercise/exercise_06/ref.py}
		
	}
	\textbf{Start:}
	
	a = [4, 5, 6]
	\linebreak
	
	\textbf{Ende:}
	
	a = [4, 5, 6]
	
	b = [4, 6]
\end{frame}

\section{Übung 2}
\begin{frame}{Klassen - Übung 2}
	\begin{itemize}
		\item[1.] Definiere eine Klasse \codeline{Fahrzeug} mit folgenden Eigenschaften: 
		\begin{itemize}
			\item Attribute: \codeline{speed}, \codeline{preis}
			\item Methode: \codeline{hupen()} soll 'Huuup' ausgeben
			\item Methode: \codeline{info()} soll einen String mit Preis und Geschwindigkeit zurückgeben (!nicht printen sondern zurückgeben) 
		\end{itemize}
		
		\item[2.] Füge eine weitere Methode \alert{fahren(dauer)} hinzu, die eine \alert{Dauer} entgegen nimmt. Sie soll die Distanz zurückgeben, die in der angegebenen Zeit zurückgelegt wurde. $(dauer * speed)$
	\end{itemize}
	Überprüfe ob die Klasse funktioniert, indem du mehrere Instanzen in einer Liste erstellst und fahren bzw. hupen lässt:
	\lstinputlisting[firstline=4, lastline=9]{../../exercise/exercise_06/class_list.py}
\end{frame}


\begin{frame}{Klassen - Übung 2}
	\begin{itemize}
		\item[3.] 
		Definiere die Klasse \codeline{PKW} und \codeline{LKW}. Beide sollen von \codeline{Fahrzeug} erben.
		Der LKW soll außerdem \codeline{hupen} überschreiben, (anderes Geräusch ausgeben).
		
		\item[4.] 
			Erweiterung der Klassen.\linebreak
			\codeline{PKW}
			\begin{itemize}
				\item \codeline{\_\_init\_\_} überschreiben, alle PKWs sollen 30.000 € kosten und 180 km/h schnell sein
				\item zusätzliche Attribute: \codeline{personen\_zahl}
				\item \codeline{info()} Methode überschreiben: soll personen\_zahl mit ausgeben
			\end{itemize}
		
			\codeline{LKW}
			\begin{itemize}
				\item \codeline{\_\_init\_\_} überschreiben, alle LKWs sollen 200.000 € kosten und 80 km/h schnell sein
				\item zusätzliche Attribute: \codeline{laderaum}
				\item \codeline{info()} Methode überschreiben: soll laderaum mit ausgeben
			\end{itemize}
	\end{itemize}
\end{frame}

\begin{frame}
	\textbf{Zusatz:} Definiere eine \codeline{Panzer} Klasse, die auch von Fahrzeug erbt. Panzer sollen 6.000.000 € kosten, 70 km/h schnell fahren und eine Anzahl von Munition haben. Außerdem soll ein Panzer \codeline{schießen} können. Er kann nur wenn Munition da ist schießen und verbraucht dabei eine Munition. Beim Schießen soll 'Bum!' ausgegeben werden.
\end{frame}

\end{document}
