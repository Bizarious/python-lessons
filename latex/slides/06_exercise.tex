% The Slide Definitions
\input{../templates/course_definitions}

% Author and Course information
% This Document contains the information about this course.

% Authors of the slides
\author{Richard Müller, Tom Felber}

% Name of the Course
\institute{Python-Kurs}

% Fancy Logo 
\titlegraphic{\hfill\includegraphics[height=1.25cm]{../templates/fsr_logo_cropped}}



% Custom Bindings
% \newcommand{\codeline}[1]{
%	\alert{\texttt{#1}}
%}


% Presentation title
% TODO Change the topic of the lesson
\title{Referenzen - Übung}
\date{18. November 2021}
\begin{document}

\maketitle

% TODO: Add your content right below here.

\begin{frame}{Referenzen - Übung 1}
	\large\lstinputlisting[firstline=1, lastline=5]{../../exercise/exercise_06/ref.py}
	a = \only<1>{?}\only<2>{\alert{[ [1, 5], [1, 5] ]}}
	
	b = \only<1>{?}\only<2>{\alert{[1, 5]}}
\end{frame}


\begin{frame}{Referenzen - Übung 2}
	\large\lstinputlisting[firstline=8, lastline=12]{../../exercise/exercise_06/ref.py}
	a = \only<1>{?}\only<2>{\alert{[2, 3]}}
\end{frame}

\begin{frame}{Referenzen - Übung 3}
	\only<1>{
		\large\lstinputlisting[firstline=25, lastline=26]{../../exercise/exercise_06/ref.py}
		
		\large\lstinputlisting[firstline=20, lastline=21]{../../exercise/exercise_06/ref.py}
	}
	\only<2>{
		\large\lstinputlisting[firstline=15, lastline=21]{../../exercise/exercise_06/ref.py}
		
	}
	\textbf{Start:}
	
	a = [4, 5, 6]
	\linebreak
	
	\textbf{Ende:}
	
	a = [4, 5, 6]
	
	b = [4, 6]
\end{frame}


\end{document}
