% The Slide Definitions
%document
\documentclass[10pt]{beamer}
%theme
\usetheme{metropolis}
% packages
\usepackage{color}
\usepackage{listings}
\usepackage[ngerman]{babel}
\usepackage[utf8]{inputenc}
\usepackage{multicol}


% color definitions
\definecolor{mygreen}{rgb}{0,0.6,0}
\definecolor{mygray}{rgb}{0.5,0.5,0.5}
\definecolor{mymauve}{rgb}{0.58,0,0.82}

\lstset{
    backgroundcolor=\color{white},
    % choose the background color;
    % you must add \usepackage{color} or \usepackage{xcolor}
    basicstyle=\footnotesize\ttfamily,
    % the size of the fonts that are used for the code
    breakatwhitespace=false,
    % sets if automatic breaks should only happen at whitespace
    breaklines=true,                 % sets automatic line breaking
    captionpos=b,                    % sets the caption-position to bottom
    commentstyle=\color{mygreen},    % comment style
    % deletekeywords={...},
    % if you want to delete keywords from the given language
    extendedchars=true,
    % lets you use non-ASCII characters;
    % for 8-bits encodings only, does not work with UTF-8
    literate={ä}{{\"a}}1 {ü}{{\"u}}1 {ö}{{\"o}}1 {Ä}{{\"A}}1 {Ü}{{\"U}}1 {Ö}{{\"O}}1 {ß}{{\ss{}}}1,
    % escapes umlauts
    frame=single,                    % adds a frame around the code
    keepspaces=true,
    % keeps spaces in text,
    % useful for keeping indentation of code
    % (possibly needs columns=flexible)
    keywordstyle=\color{blue},       % keyword style
    % morekeywords={*,...},
    % if you want to add more keywords to the set
    numbers=left,
    % where to put the line-numbers; possible values are (none, left, right)
    numbersep=5pt,
    % how far the line-numbers are from the code
    numberstyle=\tiny\color{mygray},
    % the style that is used for the line-numbers
    rulecolor=\color{black},
    % if not set, the frame-color may be changed on line-breaks
    % within not-black text (e.g. comments (green here))
    stepnumber=1,
    % the step between two line-numbers.
    % If it's 1, each line will be numbered
    stringstyle=\color{mymauve},     % string literal style
    tabsize=4,                       % sets default tabsize to 4 spaces
    % show the filename of files included with \lstinputlisting;
    % also try caption instead of title
    language = Python,
	showspaces = false,
	showtabs = false,
	showstringspaces = false,
	escapechar = ,
}

\def\ContinueLineNumber{\lstset{firstnumber=last}}
\def\StartLineAt#1{\lstset{firstnumber=#1}}
\let\numberLineAt\StartLineAt



\newcommand{\codeline}[1]{
	\alert{\texttt{#1}}
}


% Author and Course information
% This Document contains the information about this course.

% Authors of the slides
\author{Richard Müller, Tom Felber}

% Name of the Course
\institute{Python-Kurs}

% Fancy Logo 
\titlegraphic{\hfill\includegraphics[height=1.25cm]{../templates/fsr_logo_cropped}}



% Custom Bindings
% \newcommand{\codeline}[1]{
%	\alert{\texttt{#1}}
%}


% Presentation title
% TODO Change the topic of the lesson
\title{Filehandeling und Listcomprehension - Übung}
\date{13. Januar 2022}
\begin{document}

\maketitle

\begin{frame}{Gliederung}
	\setbeamertemplate{section in toc}[sections numbered]
	\tableofcontents
\end{frame}

\section{Übung 1 - Referenzen}
\begin{frame}{Übung 1 - Referenzen}
	Was macht diese Funktion ?
	\lstinputlisting[firstline=2, lastline=3]{../../exercise/exercise_09/referenzen.py}
	\only<1>{?}\only<2>{\alert{Sie gibt sich selbst aus}}
\end{frame}
\begin{frame}{Übung 1 - Referenzen}
	\lstinputlisting[firstline=6, lastline=9]{../../exercise/exercise_09/referenzen.py}
	\only<1>{x = ?}\only<2>{\alert{x = 2}}
\end{frame}
\begin{frame}{Übung 1 - Referenzen}
	\lstinputlisting[firstline=12, lastline=19]{../../exercise/exercise_09/referenzen.py}
	\only<1>{x = ?}\only<2>{\alert{x ist eine Funktion der Form f(x) = (x + 5) * 2}}
\end{frame}

\section{Übung 2 - Filehandeling / Listcomprehension}
\begin{frame}{Übung 2 - Filehandeling}
	\textbf{1.}
	
	Erstelle eine Textdatei, in der 'ab' steht. Schreibe weiterhin ein Programm, dass den Inhalt der Datein einliest, und anschließend wieder zurück in die Datei schreibt, sodass:
	\begin{itemize}
		\item alle \alert{'a'} durch \alert{'ab'}
		\item alle \alert{'b'} durch \alert{'bb'}
	\end{itemize}
	ersetzt werden.
	
	Versuche den neuen String, der zurückgeschrieben wird, in einer Zeile zu erzeugen.(list comprehension)
\end{frame}
\begin{frame}{Übung 2 - Filehandeling}
	\textbf{2.}
	
	Schreibe nun 'abc' in die Textdatei. Das Programm soll nun:
	\begin{itemize}
		\item alle \alert{'a'} durch \alert{'abc'}
		\item alle \alert{'b'} durch \alert{'bb'}
		\item alle \alert{'c'} durch \alert{'cac'}
	\end{itemize}
	ersetzen.
	Verwende ein Dictionary, um die Ersetzungsbeziehungen festzuhalten.
	
	Versuche auch diesmal, den neuen String in einer Zeile zu erzeugen. (abgesehen von der Dictionary Zeile)
	
	Probier das Programm aus, füge eigene Regeln hinzu.
	
	\textbf{Hinweis: \codeline{' '.join}}
\end{frame}

\end{document}
