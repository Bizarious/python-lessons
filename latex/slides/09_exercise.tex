% The Slide Definitions
\input{../templates/course_definitions}

% Author and Course information
% This Document contains the information about this course.

% Authors of the slides
\author{Richard Müller, Tom Felber}

% Name of the Course
\institute{Python-Kurs}

% Fancy Logo 
\titlegraphic{\hfill\includegraphics[height=1.25cm]{../templates/fsr_logo_cropped}}



% Custom Bindings
% \newcommand{\codeline}[1]{
%	\alert{\texttt{#1}}
%}


% Presentation title
% TODO Change the topic of the lesson
\title{Filehandeling und Listcomprehension - Übung}
\date{13. Januar 2022}
\begin{document}

\maketitle

\begin{frame}{Gliederung}
	\setbeamertemplate{section in toc}[sections numbered]
	\tableofcontents
\end{frame}

\section{Übung 1 - Referenzen}
\begin{frame}{Übung 1 - Referenzen}
	Was macht diese Funktion ?
	\lstinputlisting[firstline=2, lastline=3]{../../exercise/exercise_09/referenzen.py}
	\only<1>{?}\only<2>{\alert{Sie gibt sich selbst aus}}
\end{frame}
\begin{frame}{Übung 1 - Referenzen}
	\lstinputlisting[firstline=6, lastline=9]{../../exercise/exercise_09/referenzen.py}
	\only<1>{x = ?}\only<2>{\alert{x = 2}}
\end{frame}
\begin{frame}{Übung 1 - Referenzen}
	\lstinputlisting[firstline=12, lastline=19]{../../exercise/exercise_09/referenzen.py}
	\only<1>{x = ?}\only<2>{\alert{x ist eine Funktion der Form f(x) = (x + 5) * 2}}
\end{frame}

\section{Übung 2 - Filehandeling / Listcomprehension}
\begin{frame}{Übung 2 - Filehandeling / Listcomprehension}
	\textbf{1.}
	
	Erstelle eine Textdatei, in der 'ab' steht. Schreibe weiterhin ein Programm, dass den Inhalt der Datein einliest, und anschließend wieder zurück in die Datei schreibt, sodass:
	\begin{itemize}
		\item alle \alert{'a'} durch \alert{'ab'}
		\item alle \alert{'b'} durch \alert{'bb'}
	\end{itemize}
	ersetzt werden.
	
	Versuche den neuen String, der zurückgeschrieben wird, in einer Zeile zu erzeugen.(list comprehension)
\end{frame}
\begin{frame}{Übung 2 - Filehandeling / Listcomprehension }
	\textbf{2.}
	
	Schreibe nun 'abc' in die Textdatei. Das Programm soll nun:
	\begin{itemize}
		\item alle \alert{'a'} durch \alert{'abc'}
		\item alle \alert{'b'} durch \alert{'bb'}
		\item alle \alert{'c'} durch \alert{'cac'}
	\end{itemize}
	ersetzen.
	Verwende ein Dictionary, um die Ersetzungsbeziehungen festzuhalten.
	
	Versuche auch diesmal, den neuen String in einer Zeile zu erzeugen. (abgesehen von der Dictionary Zeile)
	
	Probier das Programm aus, füge eigene Regeln hinzu.
	
	\textbf{Hinweis: \codeline{' '.join}}
\end{frame}

\end{document}
