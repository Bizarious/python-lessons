% The Slide Definitions
\input{../templates/course_definitions}

% Author and Course information
% This Document contains the information about this course.

% Authors of the slides
\author{Richard Müller, Tom Felber}

% Name of the Course
\institute{Python-Kurs}

% Fancy Logo 
\titlegraphic{\hfill\includegraphics[height=1.25cm]{../templates/fsr_logo_cropped}}



% Custom Bindings
% \newcommand{\codeline}[1]{
%	\alert{\texttt{#1}}
%}


% Presentation title
\title{Filehandeling und Listcomprehension}
\date{13. Januar 2022}

\begin{document}
	
\maketitle

\begin{frame}{Gliederung}
	\setbeamertemplate{section in toc}[sections numbered]
	\tableofcontents
\end{frame}

\section*{Gesamtübersicht}
\begin{frame}{Gesamtübersicht}
	\textbf{Themen der nächsten Stunden}
	\begin{itemize}
		\item Referenzen Erklärung
		\item  Klassen
		\item Imports
		\item Nützliche Funktionen zur Iteration
		\item Lambda
		\item Unpacking
		\item \alert{File handeling}
		\item \alert{Listcomprehension}
		\item Dekoratoren
	\end{itemize}
\end{frame}

\section{Wiederholung}
\begin{frame}{Wiederholung}
	\textbf{Beim letzten Mal:}
	\begin{itemize}
		\item Unpacking
		\lstinputlisting[firstline=7,lastline=8]{resources//08_unpacking_lambda/unpacking.py}
		\lstinputlisting[firstline=19,lastline=23]{resources//08_unpacking_lambda/unpacking.py}
		\item Lambda
		\lstinputlisting[firstline=0,lastline=15]{resources//08_unpacking_lambda/lamda_syntax.py}
	\end{itemize}	
\end{frame}

\section{Filehandeling}
\begin{frame}{test}
	
\end{frame}

\section{Listcomprehension}
\begin{frame}{test}
	
\end{frame}

\end{document}
