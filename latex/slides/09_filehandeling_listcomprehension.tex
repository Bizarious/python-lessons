% The Slide Definitions
%document
\documentclass[10pt]{beamer}
%theme
\usetheme{metropolis}
% packages
\usepackage{color}
\usepackage{listings}
\usepackage[ngerman]{babel}
\usepackage[utf8]{inputenc}
\usepackage{multicol}


% color definitions
\definecolor{mygreen}{rgb}{0,0.6,0}
\definecolor{mygray}{rgb}{0.5,0.5,0.5}
\definecolor{mymauve}{rgb}{0.58,0,0.82}

\lstset{
    backgroundcolor=\color{white},
    % choose the background color;
    % you must add \usepackage{color} or \usepackage{xcolor}
    basicstyle=\footnotesize\ttfamily,
    % the size of the fonts that are used for the code
    breakatwhitespace=false,
    % sets if automatic breaks should only happen at whitespace
    breaklines=true,                 % sets automatic line breaking
    captionpos=b,                    % sets the caption-position to bottom
    commentstyle=\color{mygreen},    % comment style
    % deletekeywords={...},
    % if you want to delete keywords from the given language
    extendedchars=true,
    % lets you use non-ASCII characters;
    % for 8-bits encodings only, does not work with UTF-8
    literate={ä}{{\"a}}1 {ü}{{\"u}}1 {ö}{{\"o}}1 {Ä}{{\"A}}1 {Ü}{{\"U}}1 {Ö}{{\"O}}1 {ß}{{\ss{}}}1,
    % escapes umlauts
    frame=single,                    % adds a frame around the code
    keepspaces=true,
    % keeps spaces in text,
    % useful for keeping indentation of code
    % (possibly needs columns=flexible)
    keywordstyle=\color{blue},       % keyword style
    % morekeywords={*,...},
    % if you want to add more keywords to the set
    numbers=left,
    % where to put the line-numbers; possible values are (none, left, right)
    numbersep=5pt,
    % how far the line-numbers are from the code
    numberstyle=\tiny\color{mygray},
    % the style that is used for the line-numbers
    rulecolor=\color{black},
    % if not set, the frame-color may be changed on line-breaks
    % within not-black text (e.g. comments (green here))
    stepnumber=1,
    % the step between two line-numbers.
    % If it's 1, each line will be numbered
    stringstyle=\color{mymauve},     % string literal style
    tabsize=4,                       % sets default tabsize to 4 spaces
    % show the filename of files included with \lstinputlisting;
    % also try caption instead of title
    language = Python,
	showspaces = false,
	showtabs = false,
	showstringspaces = false,
	escapechar = ,
}

\def\ContinueLineNumber{\lstset{firstnumber=last}}
\def\StartLineAt#1{\lstset{firstnumber=#1}}
\let\numberLineAt\StartLineAt



\newcommand{\codeline}[1]{
	\alert{\texttt{#1}}
}


% Author and Course information
% This Document contains the information about this course.

% Authors of the slides
\author{Richard Müller, Tom Felber}

% Name of the Course
\institute{Python-Kurs}

% Fancy Logo 
\titlegraphic{\hfill\includegraphics[height=1.25cm]{../templates/fsr_logo_cropped}}



% Custom Bindings
% \newcommand{\codeline}[1]{
%	\alert{\texttt{#1}}
%}


% Presentation title
\title{Filehandeling und Listcomprehension}
\date{13. Januar 2022}

\begin{document}
	
\maketitle

\begin{frame}{Gliederung}
	\setbeamertemplate{section in toc}[sections numbered]
	\tableofcontents
\end{frame}

\section*{Gesamtübersicht}
\begin{frame}{Gesamtübersicht}
	\textbf{Themen der nächsten Stunden}
	\begin{itemize}
		\item Referenzen Erklärung
		\item  Klassen
		\item Imports
		\item Nützliche Funktionen zur Iteration
		\item Lambda
		\item Unpacking
		\item \alert{File handeling}
		\item \alert{Listcomprehension}
		\item Dekoratoren
	\end{itemize}
\end{frame}

\section{Wiederholung}
\begin{frame}{Wiederholung}
	\textbf{Beim letzten Mal:}
	\begin{itemize}
		\item Unpacking
		\lstinputlisting[firstline=7,lastline=8]{resources//08_unpacking_lambda/unpacking.py}
		\lstinputlisting[firstline=19,lastline=23]{resources//08_unpacking_lambda/unpacking.py}
		\item Lambda
		\lstinputlisting[firstline=0,lastline=15]{resources//08_unpacking_lambda/lamda_syntax.py}
	\end{itemize}	
\end{frame}

\section{Filehandeling}
\begin{frame}{open Befehl}
	\codeline{open()} ist eine builtin Funktion zum öffnen von Dateien
	\lstinputlisting[firstline=0,lastline=1]{resources//09_filehandeling_listcomprehension/open_basics.py}
	\codeline{open} hat verschiedene Bearbeitungsmodi
	\begin{itemize}
		\item[] \lstinputlisting[firstline=3,lastline=4]{resources//09_filehandeling_listcomprehension/open_basics.py}
		\item[] \lstinputlisting[firstline=5,lastline=6]{resources//09_filehandeling_listcomprehension/open_basics.py}
		\item[] \lstinputlisting[firstline=7,lastline=8]{resources//09_filehandeling_listcomprehension/open_basics.py}
	\end{itemize}
\end{frame}

\begin{frame}{File Objekte}
	
	Die \codeline{open} Funktion gibt ein \codeline{File} Objekt zurück. Was kann man damit machen ?
	\begin{itemize}
		\item Die Datei schließen mit \codeline{f.close()} Dies sollte immer passieren wenn die Bearbeitung abgeschlossen ist.
		\item Den kompletten Inhalt der Datei lesen mit \codeline{f.read()} 
		\item Zeile für Zeile den Inhalt lesen mit \codeline{f.readline()}
		\item In die Datei schreiben mit \codeline{f.write("content")}
		\item Die Stelle verändern an der gelesen / geschrieben wird mit \codeline{f.seek(position)}
	\end{itemize}
\end{frame}

\begin{frame}{Bearbeitungsmodi}
	Zusätzlich zu den normalen Modi gibt es den jeweiligen Modus mit einem \codeline{+}. Diese Modi haben mehr Rechte als der originale.
	
	\begin{tabular}{| m{4cm} || m{0.5cm} m{0.5cm} m{0.5cm} || m{0.5cm} m{0.5cm} m{0.5cm} |}
		\hline
		Rechte / Modus 				& r & w & a & r+ & w+ & a+ \\ \hline\hline
		Read 				& \cmark &   &   & \cmark  & \cmark  & \cmark  \\ \hline
		Write   			&   & \cmark & \cmark & \cmark  & \cmark  & \cmark  \\ \hline
		Create  			&   & \cmark & \cmark &    & \cmark  & \cmark  \\ \hline
		Am Anfang starten   & \cmark & \cmark &   & \cmark  & \cmark  &    \\ \hline
		Am Ende starten   	&   &   & \cmark &    &    & \cmark  \\ \hline
	\end{tabular}
\end{frame}

\begin{frame}{Tipp: with open()}
	Mit \codeline{with open() as name:} wird die Datei automatisch wieder geschlossen, wenn das aufgemachte scope endet. \codeline{f.close()} kann nicht vergessen werden. 
	\lstinputlisting[firstline=0,lastline=6]{resources//09_filehandeling_listcomprehension/with_open.py}
\end{frame}

\section{Listcomprehension}
\begin{frame}{test}
	
\end{frame}

\end{document}
