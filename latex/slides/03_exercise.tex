% The Slide Definitions
\input{../templates/course_definitions}

% Author and Course information
% This Document contains the information about this course.

% Authors of the slides
\author{Richard Müller, Tom Felber}

% Name of the Course
\institute{Python-Kurs}

% Fancy Logo 
\titlegraphic{\hfill\includegraphics[height=1.25cm]{../templates/fsr_logo_cropped}}



% Custom Bindings
% \newcommand{\codeline}[1]{
%	\alert{\texttt{#1}}
%}


% Presentation title
% TODO Change the topic of the lesson
\title{Booleans und Funktionen - Übung}
\date{4. November 2021}
\begin{document}

\maketitle

\begin{frame}{Gliederung}
	\setbeamertemplate{section in toc}[sections numbered]
	\tableofcontents
\end{frame}

% TODO: Add your content right below here.

\section{Booleans}

\begin{frame}{Übung 1}
	\textbf{Der letzte Versuch}
	\linebreak
	
	Es geht um den letzten Prüfungsversuch eines Studenten.
	
	Definiere 3 Variablen, die jeweils nur \alert{\texttt{True}} oder \alert{\texttt{False}} sein können: \texttt{gelernt}, \texttt{glueck} und \texttt{aufgepasst}.
	
	Wenn \textbf{(a)} eine, oder \textbf{(b)} zwei der drei Variablen \alert{\texttt{True}} sind, hat der Student bestanden und es wird \alert{\texttt{True}} ausgegeben. Implementiere diese beiden Aussagen mithilfe der dir bekannten logischen Operationen.
	
	Definiere eine weitere Variable: \texttt{beitrag\_bezahlt}. Es soll angezeigt werden, ob der Student exmatrikuliert wird. Dies hängt davon ab, ob er die Studiengebühren bezahlt und den letzten Prüfungsversuch bestanden hat. Implementiere auch diese Aussage.

\end{frame}

\section{Funktionen}

\begin{frame}{Übung 2}
	\begin{itemize}
		\item[\textbf{1.}] Definiere eine Funktion, die 3 Variablen (alles \alert{\texttt{bool}}) als Argumente nimmt und \alert{\texttt{True}} zurückgibt, wenn 2 von 3 ebenfalls \alert{\texttt{True}} sind. Probiere die Funktion aus.
		\item[\textbf{2.}] Definiere eine Funktion, die die Funktion aus \textbf{1.} mit allen möglichen Kombinationen der Argumente wiederholt aufruft.
	\end{itemize}

\end{frame}

\begin{frame}{Übung 3}
	\begin{itemize}
		\item[\textbf{1.}] Definiere eine Funktion, die eine Liste als Argument entgegennimmt. Die Funktion soll \alert{\texttt{True}} zurückgeben, wenn alle Zahlen in der Liste kleiner als \texttt{10} sind, ansonsten soll sie \alert{\texttt{False}} zurückgeben.
		\item[\textbf{2.}] Definiere eine Funktion, welche eine Liste aus Integern als erstes Argument (die Zahlen) und einen Integer als zweites Argument (der Teiler) entgegennimmt. Die Funktion soll eine neue Liste zurückgeben, die die originalen Zahlen geteilt durch den Teiler enthält. Gib dem zweiten Argument als Default-Wert die \texttt{2}.
		\begin{itemize}
			\item[\textbf{Bsp.:}] Zahlen: [5, 10, 20, 50], Teiler: 5, Ergebnis: [1, 2, 4, 10]
		\end{itemize}
		\item[\textbf{3.}] Definiere eine Funktion, die die beiden Funktionen aus \textbf{1.} und \textbf{2.} verwendet. Die Funktion soll eine Liste von Zahlen so lange durch \texttt{2} Teilen, bis alle Zahlen der Liste kleiner als \texttt{10} sind.
	\end{itemize}

\end{frame}

\end{document}
