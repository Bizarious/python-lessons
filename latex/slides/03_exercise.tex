% The Slide Definitions
\input{../templates/course_definitions}

% Author and Course information
% This Document contains the information about this course.

% Authors of the slides
\author{Richard Müller, Tom Felber}

% Name of the Course
\institute{Python-Kurs}

% Fancy Logo 
\titlegraphic{\hfill\includegraphics[height=1.25cm]{../templates/fsr_logo_cropped}}



% Custom Bindings
% \newcommand{\codeline}[1]{
%	\alert{\texttt{#1}}
%}


% Presentation title
% TODO Change the topic of the lesson
\title{Booleans und Funktionen - Übung}
\date{4. November 2021}
\begin{document}

\maketitle

\begin{frame}{Gliederung}
	\setbeamertemplate{section in toc}[sections numbered]
	\tableofcontents
\end{frame}

% TODO: Add your content right below here.

\section{Übung 1 - Booleans}

\begin{frame}{Übung 1 - Booleans}
	\textbf{Der letzte Versuch}
	\linebreak
	
	Es geht um den letzten Prüfungsversuch eines Studenten.
	
	Definiere 3 Variablen, die jeweils nur \alert{\texttt{True}} oder \alert{\texttt{False}} sein können: \texttt{gelernt}, \texttt{glueck} und \texttt{aufgepasst}.
	
	Wenn \textbf{(a)} eine, oder \textbf{(b)} zwei der drei Variablen \alert{\texttt{True}} sind, hat der Student bestanden und es wird \alert{\texttt{True}} ausgegeben. Implementiere diese Aussage mithilfe der dir bekannten logischen Operationen.
	
	Definiere eine weitere Variable: \texttt{beitrag\_bezahlt}. Es soll angezeigt werden, ob der Student exmatrikuliert wird. Dies hängt davon ab, ob er die Studiengebühren bezahlt und den letzten Prüfungsversuch bestanden hat. Implementiere auch diese Aussage.

\end{frame}

\end{document}
