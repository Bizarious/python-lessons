% The Slide Definitions
\input{../templates/course_definitions}

% Author and Course information
% This Document contains the information about this course.

% Authors of the slides
\author{Richard Müller, Tom Felber}

% Name of the Course
\institute{Python-Kurs}

% Fancy Logo 
\titlegraphic{\hfill\includegraphics[height=1.25cm]{../templates/fsr_logo_cropped}}



% Custom Bindings
% \newcommand{\codeline}[1]{
%	\alert{\texttt{#1}}
%}


% Presentation title
% TODO Change the topic of the lesson
\title{Grundlagen - Übung}
\date{\today}
\begin{document}

\maketitle

\begin{frame}{Gliederung}
	\setbeamertemplate{section in toc}[sections numbered]
	\tableofcontents
\end{frame}

% TODO: Add your content right below here.

\section{Übung 1 - Hello World}

\begin{frame}{1. Aufgabe}
	test1
\end{frame}

\section{Übung 2 - Strings}

\begin{frame}{2. Aufgabe}
	Strings, wie verbindet und formatiert man sie?
	
	Tipp: möglichst oft den code testen um früh Fehler zu finden
	
	\textbf{Schritt 1:}
	
	Mit dem \textit{input()} Befehl kann eine Eingabe aus der command line gelesen und in einer Variable abgelegt werden: \textit{var = input()}. An der Stelle, wo \textit{input()} aufgerufen wird, pausiert das Script, bis eine Eingabe gegeben wurde.
	
	Schreibe ein script, dass einen input string von der command line liest und \textit{'Hello '} ausgibt, wo bei der eingelesene String angehangen wird. Du kannst f-Strings verwenden oder die strings einfach mit + verbinden.
\end{frame}
\begin{frame}
	\textbf{Schritt 2:}
	Erweitere das Script nun um eine Startnachricht, die vor der input abfrage ausgegeben wird und nach einem Namen fragt: \textit{input("message")}. Gib eine Willkommensnachricht aus. Zum Beispiel: "Hello NAME, welcome to python" wobei NAME durch den input ersetzt werden soll. Dafür können f-Strings, die format funktion verwendet werden.
	\linebreak
	f-Strings funktionieren so: \textit{f"hallo \{'oma'\}, willkommen"} -$>$ \textit{"hallo oma, willkommen"}
	\linebreak
	
	\textbf{Schritt 3}	
	Frage nun nach dem Vor und Nachnamen seperat und baue die Werte in den Ausgabestring ein.
\end{frame}


\section{Übung 3 - String-Methoden}

\begin{frame}{3. Aufgabe}
	\textbf{Schritt 1}
	
	Mit der Dokumentation auf dieser Seite
	\href{https://docs.python.org/3/library/stdtypes.html\#string-methods}{https://docs.python.org/3/library/stdtypes.html\#string-methods}
	Schreibe ein script, dass einen Input-String annimmt und prüft ob folgende bedingungen wahr sind:
	
	\begin{itemize}
		\item[(1.)] der String ist großgeschrieben
		\item[(2.)] der String besteht aus Zahlen
		\item[(3.)] im String kommt diese sequenz vor: ':-)'
		\item[(4.)] keine der vorherigen Bedingungen ist wahr
	\end{itemize}
	
	Gib aus welcher der Fälle eingetreten ist. Wenn mehrere zutreffend sind, reicht es einen anzugeben. \linebreak
	Substrings erkennen: mit \textit{in} kann nach Teilelementen und auch Teilstrings gesucht werden. \linebreak
	Python hat kein switch case statement, nur if, else, elif (statt else if).
\end{frame}
\begin{frame}
	\textbf{Schritt 2}
	\linebreak
	Gib im 1. Fall, wenn der String großgeschrieben ist, zusätzlich den String als kleingeschriebene Variante aus.
	\linebreak
	
	\textbf{Schritt 3}
	\linebreak
	Im 2. Fall, wenn eine Zahl eingegeben wurden, lass das Script nach einer weiteren fragen und gib die Summe aus.
	\linebreak
	Der Konstruktor eines Typs kann benutzt werden um einen anderen Wert in diesen Typ umzuwandeln. z.B. str(), int()
	\linebreak
\end{frame}
\begin{frame}
	\textbf{Schritt 4}
	\linebreak
	Im 4. Fall, wenn kein anderer Fall eingetreten ist, trenne den Eingabestring an jedem 'a', und füge die Teile wieder zusammen, und ergänze ein semicolon zwischen jedem der Teile.
	\linebreak
	Strings können mit \textit{split(separator)} an einem Septerator getrennt werden.
	\linebreak
	Und mit \textit{separator\_string.join(['list', 'of', 'strings'])} zusammengefügt werden.
\end{frame}

\end{document}
