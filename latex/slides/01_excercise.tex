% The Slide Definitions
\input{../templates/course_definitions}

% Author and Course information
% This Document contains the information about this course.

% Authors of the slides
\author{Richard Müller, Tom Felber}

% Name of the Course
\institute{Python-Kurs}

% Fancy Logo 
\titlegraphic{\hfill\includegraphics[height=1.25cm]{../templates/fsr_logo_cropped}}



% Custom Bindings
% \newcommand{\codeline}[1]{
%	\alert{\texttt{#1}}
%}


% Presentation title
% TODO Change the topic of the lesson
\title{Grundlagen - Übung}
\date{\today}


\begin{document}

\maketitle

\begin{frame}{Gliederung}
	\setbeamertemplate{section in toc}[sections numbered]
	\tableofcontents
\end{frame}

% TODO: Add your content right below here.

\section{Übung 1 - Hello World}

\begin{frame}{1. Aufgabe}
	test1
\end{frame}

\section{Übung 2 - Strings}

\begin{frame}{2. Aufgabe}
	Strings, wie verbindet und formatiert man sie?
	
	Tipp: möglichst oft den code testen um früh Fehler zu finden
	
	\textbf{Schritt 1:}
	
	Mit dem \textit{input()} Befehl kann eine Eingabe aus der command line gelesen und in einer Variable abgelegt werden: \textit{var = input()}. An der Stelle, wo \textit{input()} aufgerufen wird, pausiert das Script, bis eine Eingabe gegeben wurde.
	
	Schreibe ein script, dass einen input string von der command line liest und \textit{'Hello '} ausgibt, wo bei der eingelesene String angehangen wird. Du kannst f-Strings verwenden oder die strings einfach mit + verbinden.
\end{frame}
\begin{frame}
	\textbf{Schritt 2:}
	Erweitere das Script nun um eine Startnachricht, die vor der input abfrage ausgegeben wird und nach einem Namen fragt: \textit{input("message")}. Gib eine Willkommensnachricht aus. Zum Beispiel: "Hello NAME, welcome to python" wobei NAME durch den input ersetzt werden soll. Dafür können f-Strings, die format funktion verwendet werden.
	\linebreak
	f-Strings funktionieren so: \textit{f"hallo \{'oma'\}, willkommen"} -$>$ \textit{"hallo oma, willkommen"}
	\linebreak
	
	\textbf{Schritt 3}	
	Frage nun nach dem Vor und Nachnamen seperat und baue die Werte in den Ausgabestring ein.
\end{frame}


\section{Übung 3 - String-Methods}

\begin{frame}{3. Aufgabe}
	test3
\end{frame}


\end{document}
