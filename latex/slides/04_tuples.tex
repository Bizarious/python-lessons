% The Slide Definitions
%document
\documentclass[10pt]{beamer}
%theme
\usetheme{metropolis}
% packages
\usepackage{color}
\usepackage{listings}
\usepackage[ngerman]{babel}
\usepackage[utf8]{inputenc}
\usepackage{multicol}


% color definitions
\definecolor{mygreen}{rgb}{0,0.6,0}
\definecolor{mygray}{rgb}{0.5,0.5,0.5}
\definecolor{mymauve}{rgb}{0.58,0,0.82}

\lstset{
    backgroundcolor=\color{white},
    % choose the background color;
    % you must add \usepackage{color} or \usepackage{xcolor}
    basicstyle=\footnotesize\ttfamily,
    % the size of the fonts that are used for the code
    breakatwhitespace=false,
    % sets if automatic breaks should only happen at whitespace
    breaklines=true,                 % sets automatic line breaking
    captionpos=b,                    % sets the caption-position to bottom
    commentstyle=\color{mygreen},    % comment style
    % deletekeywords={...},
    % if you want to delete keywords from the given language
    extendedchars=true,
    % lets you use non-ASCII characters;
    % for 8-bits encodings only, does not work with UTF-8
    literate={ä}{{\"a}}1 {ü}{{\"u}}1 {ö}{{\"o}}1 {Ä}{{\"A}}1 {Ü}{{\"U}}1 {Ö}{{\"O}}1 {ß}{{\ss{}}}1,
    % escapes umlauts
    frame=single,                    % adds a frame around the code
    keepspaces=true,
    % keeps spaces in text,
    % useful for keeping indentation of code
    % (possibly needs columns=flexible)
    keywordstyle=\color{blue},       % keyword style
    % morekeywords={*,...},
    % if you want to add more keywords to the set
    numbers=left,
    % where to put the line-numbers; possible values are (none, left, right)
    numbersep=5pt,
    % how far the line-numbers are from the code
    numberstyle=\tiny\color{mygray},
    % the style that is used for the line-numbers
    rulecolor=\color{black},
    % if not set, the frame-color may be changed on line-breaks
    % within not-black text (e.g. comments (green here))
    stepnumber=1,
    % the step between two line-numbers.
    % If it's 1, each line will be numbered
    stringstyle=\color{mymauve},     % string literal style
    tabsize=4,                       % sets default tabsize to 4 spaces
    % show the filename of files included with \lstinputlisting;
    % also try caption instead of title
    language = Python,
	showspaces = false,
	showtabs = false,
	showstringspaces = false,
	escapechar = ,
}

\def\ContinueLineNumber{\lstset{firstnumber=last}}
\def\StartLineAt#1{\lstset{firstnumber=#1}}
\let\numberLineAt\StartLineAt



\newcommand{\codeline}[1]{
	\alert{\texttt{#1}}
}


% Author and Course information
% This Document contains the information about this course.

% Authors of the slides
\author{Richard Müller, Tom Felber}

% Name of the Course
\institute{Python-Kurs}

% Fancy Logo 
\titlegraphic{\hfill\includegraphics[height=1.25cm]{../templates/fsr_logo_cropped}}



% Custom Bindings
% \newcommand{\codeline}[1]{
%	\alert{\texttt{#1}}
%}


% Presentation title
\title{Tuples}
\date{4. November 2021}

\begin{document}
	
\maketitle

\begin{frame}{Gliederung}
	\setbeamertemplate{section in toc}[sections numbered]
	\tableofcontents
\end{frame}

\section{Wiederholung}
\begin{frame}{Wiederholung}
	
\end{frame}

\section{Tuples}
\begin{frame}{Tuples}
	Tuple ist ein Datentyp, der ähnlich der Liste ist.
	\begin{center}
		\begin{tabular}{m{6cm} | m{2cm} | m{2cm}}
			\hline\hline
			Eigenschaft & Liste & Tuple \\
			\hline\hline
			enthält mehrere Elemente & \cmark & \cmark \\ 
			\pause
			geordnet & \cmark & \cmark \\ \pause
			kann mit \codeline{[i]} indexiert werden (Elemente herausgreifen) & \cmark & \cmark \\ \hline
			\pause
			Elemente können angehangen / gelöscht werden & \cmark & \xmark \\ \pause
			Werte von Elementen können neu zugewiesen werden & \cmark & \xmark \\
			\pause
			Referenzverhalten & mutable & immutable \\
		\end{tabular}
	\end{center}
\end{frame}

\begin{frame}{Tupel - Beispiele}
	Initialisierung ähnlich der Liste, \codeline{()} statt \codeline{[]}
	\lstinputlisting[firstline=1, lastline=2]{resources/04tuples/tuples.py}
	\pause
	Einzelne Elemente herausgreifen genau wie bei Listen. Slices funktionieren auch.
	\lstinputlisting[firstline=5, lastline=8]{resources/04tuples/tuples.py}
	\pause
	Eine Liste zu einem Tuple umwandeln
	\lstinputlisting[firstline=21, lastline=23]{resources/04tuples/tuples.py}
\end{frame}
\begin{frame}{Tupel - Beispile}
	Aber! Elemente neu zu setzen ist nicht möglich:
	\lstinputlisting[firstline=10, lastline=10]{resources/04tuples/tuples.py}
	\pause
	Nur das gesamte Tuple kann neu gesetzt werden:
	\lstinputlisting[firstline=12, lastline=12]{resources/04tuples/tuples.py}
\end{frame}

\begin{frame}{Tuples - Warum ?}
	Warum Tuples benutzen wenn Listen mehr können ? Es kommt auf den Fall an.
	\linebreak\linebreak
	\textbf{Tuples sind gut für:}
	\begin{itemize}
		\pause
		\item statische Daten: verhindern das ausversehen Daten verändert werden die als Konstanten gemeint sind
		\pause
		\item vielen Daten (die keine Listenfunktionialität benötigen): werden schneller verarbeitet als Listen
		\pause
		\item als Keys für Dictionaries: da sie immutable sind können Tuples als Key verwendet werden (Listen nicht)
	\end{itemize}
\end{frame}	

\end{document}
