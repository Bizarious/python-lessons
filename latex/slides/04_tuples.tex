% The Slide Definitions
\input{../templates/course_definitions}

% Author and Course information
% This Document contains the information about this course.

% Authors of the slides
\author{Richard Müller, Tom Felber}

% Name of the Course
\institute{Python-Kurs}

% Fancy Logo 
\titlegraphic{\hfill\includegraphics[height=1.25cm]{../templates/fsr_logo_cropped}}



% Custom Bindings
% \newcommand{\codeline}[1]{
%	\alert{\texttt{#1}}
%}


% Presentation title
\title{Tupel und Dictionaries}
\date{11. November 2021}

\begin{document}
	
\maketitle

\begin{frame}{Gliederung}
	\setbeamertemplate{section in toc}[sections numbered]
	\tableofcontents
\end{frame}

\section{Wiederholung}
\begin{frame}{Wiederholung}
	\textbf{Beim letzten Mal}
	\begin{itemize}
		\item Booleans mit \codeline{and or not}
		\item Funktionen \lstinputlisting[firstline=14,lastline=18]{resources/03bool_fun_dict/func.py}
		\item Selsames Verhalten von Listen im Vergleich zu Zahlen
	\end{itemize}
\end{frame}


\section{Tupel}
\begin{frame}{Tupel}
	Der Tupel (\codeline{tuple}) ist ein Datentyp, der ähnlich der Liste ist.
	\begin{center}
		\begin{tabular}{m{6cm} | m{2cm} | m{2cm}}
			\hline\hline
			Eigenschaft & Liste & Tuple \\
			\hline\hline
			enthält mehrere Elemente & \cmark & \cmark \\ 
			
			geordnet & \cmark & \cmark \\ 
			kann mit \codeline{[i]} indexiert werden (Elemente herausgreifen) & \cmark & \cmark \\ \hline
			
			Elemente können angehangen / gelöscht werden & \cmark & \xmark \\ 
			Werte von Elementen können neu zugewiesen werden & \cmark & \xmark \\
			
			Referenzverhalten & mutable & immutable \\
		\end{tabular}
	\end{center}
\end{frame}

\begin{frame}{Tupel - Beispiele}
	Initialisierung ähnlich der Liste, \codeline{()} statt \codeline{[]}
	\lstinputlisting[firstline=1, lastline=2]{resources/04tuples/tuples.py}
	
	Einzelne Elemente herausgreifen, genau wie bei Listen. Slices funktionieren auch.
	\lstinputlisting[firstline=5, lastline=8]{resources/04tuples/tuples.py}
	
	Eine Liste zu einem Tuple umwandeln
	\lstinputlisting[firstline=21, lastline=23]{resources/04tuples/tuples.py}
\end{frame}
\begin{frame}{Tupel - Beispile}
	Aber! Elemente neu zu setzen ist nicht möglich:
	\lstinputlisting[firstline=10, lastline=10]{resources/04tuples/tuples.py}
	
	Nur das gesamte Tuple kann neu gesetzt werden:
	\lstinputlisting[firstline=12, lastline=12]{resources/04tuples/tuples.py}
\end{frame}

\begin{frame}{Tuples - Warum ?}
	Warum Tuples benutzen wenn Listen mehr können ? Es kommt auf den Fall an.
	\linebreak\linebreak
	\textbf{Tuples sind gut für:}
	\begin{itemize}
		
		\item statische Daten: verhindern, dass ausversehen Daten verändert werden, die als Konstanten gemeint sind
		
		\item vielen Daten (die keine Listenfunktionialität benötigen): werden schneller verarbeitet als Listen
		
		\item als Keys für Dictionaries: da sie immutable sind, können Tuples als Key verwendet werden (Listen nicht)
	\end{itemize}
\end{frame}	

\section{Dictionaries}
\begin{frame}{Dictionaries - Einleitung}
	\includegraphics[width=\linewidth]{resources/03bool_fun_dict/lexikon.jpeg}
\end{frame}

\begin{frame}{Dictionaries - Erstellen / Lesen / Löschen}
	In Dictionaries können \alert{\texttt{Key - Value}} Paare abgelegt werden.
	\linebreak
	Die Keys müssen dabei einzigartig sein. Die Values nicht.
	\linebreak
	Keys werden links, Values rechts angegeben.
	\lstinputlisting[lastline=4]{resources/03bool_fun_dict/dicts.py}
	
	Mit dem Key kann der zugehörige Wert gelesen werden.
	\lstinputlisting[firstline=6,lastline=7]{resources/03bool_fun_dict/dicts.py}
	
	Oder gelesen und gleichzeitig gelöscht werden.
	\lstinputlisting[firstline=31,lastline=32]{resources/03bool_fun_dict/dicts.py}
	
\end{frame}

\begin{frame}{Dictionaries - Paare hinzufügen / überschreiben}
	Die Syntax zum hinzufügen kann genauso benutzt werden, um einen vorhandenen Key zu überschreiben.
	\lstinputlisting[firstline=4,lastline=4]{resources/03bool_fun_dict/dicts.py}
	\lstinputlisting[firstline=8,lastline=11]{resources/03bool_fun_dict/dicts.py}
\end{frame}

\begin{frame}{Dictionaries - Paare hinzufügen / überschreiben}
	Die meisten Typen können als Key benutzt werden. Listen und Dictionaries bilden hierzu eine Ausnahme, sie sind als Key nicht zulässig.
	\lstinputlisting[firstline=23,lastline=28]{resources/03bool_fun_dict/dicts.py}
\end{frame}

\end{document}
