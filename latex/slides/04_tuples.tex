% The Slide Definitions
\input{../templates/course_definitions}

% Author and Course information
% This Document contains the information about this course.

% Authors of the slides
\author{Richard Müller, Tom Felber}

% Name of the Course
\institute{Python-Kurs}

% Fancy Logo 
\titlegraphic{\hfill\includegraphics[height=1.25cm]{../templates/fsr_logo_cropped}}



% Custom Bindings
% \newcommand{\codeline}[1]{
%	\alert{\texttt{#1}}
%}


% Presentation title
\title{Tuples}
\date{4. November 2021}

\begin{document}
	
\maketitle

\begin{frame}{Gliederung}
	\setbeamertemplate{section in toc}[sections numbered]
	\tableofcontents
\end{frame}

\section{Wiederholung}
\begin{frame}{Wiederholung}
	
\end{frame}

\section{Tuples}
\begin{frame}{Tuples}
	Tuple ist ein Datentyp, der ähnlich der Liste ist.
	\begin{center}
		\begin{tabular}{m{6cm} | m{2cm} | m{2cm}}
			\hline\hline
			Eigenschaft & Liste & Tuple \\
			\hline\hline
			enthält mehrere Elemente & \cmark & \cmark \\ 
			\pause
			geordnet & \cmark & \cmark \\ \pause
			kann mit \codeline{[i]} indexiert werden (Elemente herausgreifen) & \cmark & \cmark \\ \hline
			\pause
			Elemente können angehangen / gelöscht werden & \cmark & \xmark \\ \pause
			Werte von Elementen können neu zugewiesen werden & \cmark & \xmark \\
			\pause
			Referenzverhalten & mutable & immutable \\
		\end{tabular}
	\end{center}
\end{frame}

\begin{frame}{Tuples - Beispiele}
	Initialisierung ähnlich der Liste, \codeline{()} statt \codeline{[]}
	\lstinputlisting[firstline=1, lastline=2]{resources/04tuples/tuples.py}
	\pause
	Einzelne Elemente herausgreifen genau wie bei Listen. Slices funktionieren auch.
	\lstinputlisting[firstline=5, lastline=8]{resources/04tuples/tuples.py}
	\pause
	Aber! Elemente neu zu setzen ist nicht möglich:
	\lstinputlisting[firstline=10, lastline=10]{resources/04tuples/tuples.py}
	\pause
	Nur das gesamte Tuple kann neu gesetzt werden:
	\lstinputlisting[firstline=12, lastline=12]{resources/04tuples/tuples.py}
\end{frame}

\begin{frame}{Tuples - Warum ?}
	Warum Tuples benutzen wenn Listen mehr können ?
\end{frame}	

\end{document}
